\chapter{Conclusion}
\label{chap:conclusion}

The magnitude of user data generated and stored on computing devices
and with third party providers increases ever day. So too increases
the range of threats to which such data is subjected.  Building
systems that improve the privacy and security of such user data is
therefore of critical importance.

Unfortunately, many traditional privacy and security enhancing systems
are not well adapted to modern usage practices, leading users to
forego the use of such systems all together. Privacy and security
enhancing systems must not merely protect user data, they must do so
in a manner that preserves a user's ability to use the devices and
services they wish in the manner to which they are accustomed. Failure
to do so leads to the creation of systems that go underutilized or
unused.

One of the major hurdles to building privacy and security enhancing
technologies lies in the challenges associated with securing the
secrets needed to bootstrap the secure operation of such
technologies. Such secrets range from the encryption keys necessary to
encrypt user data to the passwords necessary to access online
services. Coupled with the need to support desirable use cases, this
``secret storage challenge'' makes it difficult to build effective
security and privacy enhancing systems.

The Secret Storage as a Service (SSaaS) model provides a mechanism for
securely storing user secrets in a manner that supports a wide range
of use cases. The SSaaS system also provides mechanisms for reducing
the trust users must place in third parties, making such trust a
severable and tradable commodity in a market favoring increased
trustworthiness. The SSaaS model maximizes a user's ability to
flexibly control the manner in which their secrets are used (and by
proxy, the manner in which any data such secrets protect is used).

This dissertation makes several key contributions to the state of the
art. These include:

\begin{packed_item}
\item A model for and analysis of third party trust
  (Chapter~\ref{chap:trust}).
\item An overview of the SSaaS model and the privacy and security
  enhancing benefits it can provide (Chapter~\ref{chap:ssaas}).
\item A discussion of how the SSaaS model can be integrated to improve
  the security and privacy of a range of user- and developer-facing
  applications (Chapter~\ref{chap:apps}).
\item Custos -- a first generation SSaaS prototype
  (Chapter~\ref{chap:custos}).
\item Tutamen -- a next generation SSaaS prototype with robust support
  for sharding secrets across multiple storage providers and
  autonomous operation (Chapter~\ref{chap:tutamen}).
\item A discussion of the policy implications necessary to build and
  support a healthy SSaaS ecosystem (Chapter~\ref{chap:policy}).
\end{packed_item}

While this work represents a notable exploration of the SSaaS idea, it
is neither exhaustive nor complete. Future work related to this topic
might include the study of how the auditing components for the SSaaS
model can be leveraged to automate various components of SSaaS
security. Such automation could serve to further decrease usage
burdens will increasing user security. The Tutamen work presented
herein would also benefit from further development to increase its
performance capabilities and make it a fully production-ready
reference platform. Finally, there is much additional policy work to
be done, from efforts aimed at standardizing and encouraging the
adoption of an SSaaS protocol to ongoing defenses of cryptography and
other core security-enhancing technologies.

The SSaaS model can improve the security and privacy of computing
systems across a range of use cases. These improvements will help to
ensure that the computing services that have provided so many benefits
over the previous 20 years continue to provide benefits for the next
20 years. Flexible secret storage systems such as those proposed
herein are a critical component of future secure computing systems.

%%  LocalWords:  SSaaS Custos Tutamen
