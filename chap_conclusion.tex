\chapter{Conclusion}
\label{chap:conclusion}

The sheer magnitude of user data generated and stored on computing
devices and with third party providers increases ever day. So to
increases the range of threats to which such data is subjected.
Building systems that increase the privacy and security of such user
data is thus of critical importance.

Unfortunately, many such traditional privacy and security enhancing
systems are not well adapted to modern usage practices, leading to
users to forego the use of such systems all together. Privacy and
security enhancing systems must not merely protect user data, they
must do so in a manner that preserves a user's ability to use the
devices and services they wish in the manner to which they are
accustomed. Failure to do so leads to the creation of technologies
that go underutilized or unused.

One of the major hurtles to building privacy and security enhancing
technologies lies in the challenges associated with securing the
secrets necessary to bootstrap secure operation of such
technologies. Such secrets range from the encryption keys necessary to
encrypt user data to the password necessary to access online
services. Coupled with the need to support desirable use cases, the
secret storage challenge makes it difficult to build effective
security and privacy enhancing systems.

The Secret Storage as a Service (SSaaS) model provides a mechanism for
securely storing user secrets in a manner that supports a wide range
of use cases. The SSaaS system also provides mechanisms for reducing
the trust users must place in third parties: making such trust a
severable and tradable commodity on a market favoring increased third
party trustworthiness. The SSaaS model also maximizes a user's
flexibility to control the manner in which their secrets are used (and
by proxy, the manner in which any data such secrets are used to
protect is used).

This dissertation makes several key contributions to the state of the
are. These include:

\begin{packed_item}
\item A model for and analysis of third party trust
  (Chapter~\ref{chap:trust}).
\item An overview of the SSaaS model and the privacy and security
  enhancing benefits it can provide (Chapter~\ref{chap:ssaas}).
\item A discussion of how the SSaaS model can be integrated to improve
  the security and privacy of a range of user and developer facing
  applications (Chapter~\ref{chap:apps}).
\item Custos -- a first generation SSaaS prototype
  (Chapter~\ref{chap:custos}).
\item Tutamen -- a next generation SSaaS prototype with robust support
  for multi-secret storage provider use cases and autonomous operation
  (Chapter~\ref{chap:tutamen}).
\item A discussion of the policy implications necessary to build and
  support a healthy SSaaS ecosystem
\end{packed_item}

The SSaaS model and related work presented in this document can be
used to increase the security and privacy of user data across a range
of use cases. Such a security increase benefits the public at large,
and helps to ensure that the computing services and capabilities that
have provided so many benefits over the previous 20 years continue to
do so for the next 20 years. Such solutions are a critical component
of future computing systems.

%%  LocalWords:  SSaaS Custos Tutamen
