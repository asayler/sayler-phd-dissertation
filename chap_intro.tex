\chapter{Introduction}
\label{chap:intro}

\section{Overview}

Over the last decade, computing has undergone a monumental shift from
locally stored data on a single personal computer to cloud-based data
storage on a multitude of third party servers. This shift has
generated many benefits: sharing data with other users is trivial,
multi-modal communication between users is easy, and compute devices
are largely ephemeral, easily replaced or transitioned between without
any significant overhead or loss of user data. This transition,
however, has a significant side effect: user data is now largely
stored in manners where it is easily accessible to third parties
beyond the user's immediate control. The shift from locally stored and
controlled user data to third-party user data has a number of
increasingly clear consequences, from increased risk of data
compromise by hackers targeting centralized cloud data stores, to
reduced legal protections from government introspection, to the use of
user statistics in ``big-data'' systems capable of ascertaining more
private information than ever before.

The popularity of the cloud model leads one to believe that most users
are willing to trade the privacy and control afforded by local storage
for the convenience and features cloud-based services provide. Never
the less, a 2014 Pew Research study found that over 90\% of American
adults agree that they have lost control over the data they store in
the cloud, 80\% are concerned about how cloud companies are using
their data, and 70\% are concerned about the manner is which the
government might access their data in the
cloud~\cite{pew-privsec14}. Furthermore, the range of both recently
publicized data leaks at large companies
(e.g.~\cite{apple-icloudleak}) as well as ongoing government
intrusions into cloud-based user data stores
(e.g.~\cite{GreenwaldPrism}) has propelled the debate over user
privacy in the age of the cloud to new levels.

The traditional viewpoint holds that users must choose between either
the conveniences the cloud provides or the privacy and security of
locally stored data. I do not feel that this is true. Instead, I
believe that there are mechanisms that can allow users to retain a
high degree of control over how their data is stored, accessed, and
used while still leveraging a variety of modern third-party
services. The key is disentangling the service we wish to use on the
basis of the features they provide from the third parties we must
trust with control over and access to our data.

To achieve this disentanglement, I propose a new paradigm called
Secret Storage as a Service (SSaaS). In an SSaaS ecosystem, a user
designates one or more trusted SSaaS providers (either self hosted or
third party) with storing and regulating access to their private
secrets (personal information, encryption keys, etc) on their
behalf. Existing technologies and services can than interface with
these SSaaS providers via a standard interface to access user secrets
as allowed by a user-defined set of access control rules. I will
discuss several benefits to this arrangement over the existing
practice of selecting third party services on the basis of their
feature set and implicit providing the same providers with unfettered
access to user data. In particular:

\begin{packed_desc}
\item[No Single Trusted Third Party] \hfill \\ In an SSaaS ecosystem,
  the secret storage provider (SSP) is separate from provider of the
  end-user cloud service (e.g. Dropbox, Gmail, etc) or device
  (e.g. Apple, Lenovo, etc). Furthermore, a user may shard their
  secrets across multiple SSPs, or even host their own SSP. This
  ensures that a user is not giving any single entity control over or
  unfettered access to their data.
\item[Separation of Duties] \hfill \\
  In an SSaaS ecosystem, a user selects a secret storage provider on
  the basis of their trust in that provider while selecting a cloud
  service provider on the basis of the end-user features they
  provide. This allows a user to optimize each selection individually
  instead of having to chose a single provider on the basis of both
  trust and feature set, likely having to sacrifice one in favor of
  the other.
\item[Support for Existing Use Cases] \hfill \\ The SSaaS ecosystem is
  capable of supporting many modern use cases such as sharing data
  with other users or syncing it across a number of personal computing
  devices. Thus, SSaaS allows users to gain privacy and security
  benefits without having to forgo common and popular use cases.
\end{packed_desc}

\section{Motivating Examples}

As a motivating example, consider the Dropbox cloud file locker
service~\cite{dropbox}. Dropbox provides a service through which users
may upload arbitrary files in order to sync them between multiple
devices and to share them with other users. In order to support this
functionally, Dropbox stores a copy of each user's uploaded files on
the Dropbox servers. This ensures that Dropbox can provide copies of
the files to new user devices or to other users when asked to sync or
share a file on the user's behalf.

But how Private are a user's files once uploaded to Dropbox? While
Dropbox does encrypt files while they are stored on the Dropbox
servers as well as while they are in transit between the Dropbox
servers and a client machine~\cite{dropbox-security}, Dropbox also
holds a copy of the associated encryption keys, enabling them to
decrypt a user's files whenever they desire. This also means that an
adversary may gain access to the cleartext user files if they are able
to compromise Dropbox's servers. The government could also access
cleartext user files should a court or other entity compel Dropbox to
provide both the files and the associated encryption keys. Clearly
Dropbox's practice of storing both a user's encrypted files as well as
a copy of the associated encryption keys provides only marginally more
security and privacy of the user data then not using encryption at
all.

An alternative approach would be for Dropbox to put the user in charge
of encrypting/decrypting files and storing all necessary encryption
keys, ensuring the Dropbox itself never has direct access to
unencrypted user files. While this form of client-side encryption
could potentially increase the privacy and security of user data in
the event that Dropbox's data stores are compromised, searched,
monitored, or simply misused, is also has some significant downsides:

\begin{packed_enum}
\item It breaks Dropbox's sharing use case. While user's can still
  share encrypted version of their files, they would then have to
  exchange the associated encryption key out of band in order to
  effectively decrypt, read, or update any shared file. This
  essentially nullifies Dropbox's appeal as a simple method for
  sharing files with other users.
\item It complicates Dropbox's syncing use case. Whereas before a
  Dropbox user can bootstrap a new Dropbox client device simply by
  signing into their Dropbox account, users must now both sign into
  their Dropbox account and provide a copy of their encryption
  key/keys in order for the Dropbox client to successfully perform the
  required client-side encryption operations. This adds an additional
  step to the Dropbox setup process, potentially driving away novice
  and lay-users.
\item If the user ever losses their encryption keys, they will
  effectively lose access to all of their Dropbox-stored
  files. Similarly, if a user mishandles their keys in a manner that
  allows others to access them, they have effectively negated the
  additional privacy or security client-side encryption provides. A
  user would have to be diligent about ensuring they maintain access
  to their keys via backups, etc, while also ensuring their keys do
  not fall into the wrong hands. Again, this is a non-trivial burden
  for many users.
\end{packed_enum}

Thus, we're left in a situation where the user must chose between the
convenience of using Dropbox as it exists today while also sacrificing
a significant degree of privacy over the files they upload to Dropbox
or the burden of traditional client-side encryption models where the
trust they must place in Dropbox is reduced, but where many core
Dropbox uses cases (e.g. sharing) are also no longer feasible and the
burden of using Dropbox is significantly increased. Neither of these
are ideal solutions. We would like a solution that allows the user to
leverage the existing convenience and benefits of using Dropbox while
also reducing the trust they must place in the Dropbox corporation (or
those with power over it).

These challenges are not unique to Dropbox. There are many modern
technologies and services that force the user to chose between
convenience of use and feature set or privacy and control of their
data. For example:

\begin{packed_desc}
\item[Mobile Computing Devices] \hfill \\ Phones, tablets, and laptops
  have become ubiquitous modes of modern computing, storing large
  fractions of our personal data and carrying out computations on our
  behave. But these devices, while convenient, are also prone to loss,
  theft, and remote exploitation, exposing the data they store and
  computations they undertake to a range of external actors.
\item[Cloud Computing Infrastructure] \hfill
  \\ Infrastructure-as-a-Service (IaaS) systems such as Amazon's
  EC2~\cite{amazon-ec2} or Google's Compute
  Engine~\cite{google-compute} are popular mechanisms for hosting
  modern compute services. Unfortunately these services require the
  user to fully trust the backing infrastructure provider and make it
  difficult to deploy security-enhancing systems like full disk
  encryption due to the user's lack of physical server access.
\item[User Account Registration] \hfill \\ We're constantly being asked to
  register for services available online. This means proving the same
  identity-confirming personal data to third party after third party
  with little ability to police how this data is stored or used after
  it is provided.
\end{packed_desc}

All of these examples share a common deficiency: they force the user
into a position of choosing between desirable feature sets or desirable
security and privacy qualities. It is this deficiency that I seek to
quantify and resolve.

\section{Background}

This work builds on a number of established topics related to
computing, privacy, and security. I touch on the basics of each in the
remainder of this section. Chapter~\ref{chap:related} touches on
further related work.

\subsection{Cryptography}

Many of the topics discussed in this proposal leverage cryptographic
primitives as the basis of various security and privacy
guarantees. This is largely because it represent a security primitive
that does not rely on trusting specific people, platforms, or systems
in order to securely function. Instead, it requires that we place our
trust in only one thing: the underlying math. This has led to the
proliferation of cryptography as the security primitive on which many
other security features are built.

\subsection{Usability}

\subsection{Storage}

\subsection{The Cloud}

\subsection{Privacy and Security}

%%  LocalWords:  SSaaS SSP Lenovo SSPs IaaS
