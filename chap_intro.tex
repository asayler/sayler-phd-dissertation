\chapter{Introduction}
\label{chap:intro}

\begin{chapquote}{William S. Burroughs, \textit{Friend Magazine, 1970}}
  ``A paranoid man is a man who knows a little about what's going on.''
\end{chapquote}

\section{Overview}
\label{chap:intro:overview}

Over the last decade, computing has undergone a monumental shift from
locally stored data on a single personal computer to cloud-based data
stored on a multitude of third party servers. This shift has generated
many benefits: sharing data with other users is trivial, multi-modal
communication between users is easy, and compute devices are largely
ephemeral, easily replaced or transitioned between without any
significant overhead or loss of user data. This transition, however,
has a significant side effect: user data is now largely stored in a
manner where it is easily accessible to third parties beyond the
user's immediate control. The shift from locally stored and controlled
data to third-party stored and controlled data has a number of
consequences, from increased risk of data compromise by hackers
targeting centralized third-party data stores, to reduced legal
protections from government introspection, to its use in ``big-data''
systems capable of ascertaining more private information than most
users likely intend to share.

The popularity of the cloud model leads one to believe that most users
are willing to trade the privacy and control afforded by traditional
local compute and storage for the convenience and features cloud-based
services provide. And yet, a 2014 Pew Research study found that over
90\% of American adults agree that they have lost control over the
data they store in the cloud, 80\% are concerned about how cloud
companies are using their data, and 70\% are concerned about the
manner is which the government might access their data in the
cloud~\cite{pew-privsec14}. Furthermore, the myriad of recently
publicized data leaks at large companies
(e.g.~\cite{apple-icloudleak}) as well as ongoing government
intrusions into third-party user data stores
(e.g.~\cite{greenwald-prism}) has propelled the debate over user
privacy in the age of the cloud to new levels.

The traditional viewpoint holds that users must choose between either
the conveniences the cloud provides or the privacy and security of
locally stored and processed data. This document aims to refute that
idea. Instead, the method proposed herein aim to allow users to retain
a high degree of control over how their data is stored, accessed, and
used while still allowing them to leverage a variety of modern
third-party services. The solution lies in developing systems that
allow users to place limits on the degree to which they must trust any
single third party while still allowing them to leverage the desirable
features such parties provide.

To achieve such a solution, this document proposes a new data storage
model called ``Secret Storage as a Service'' (SSaaS). In an SSaaS
ecosystem, a user designates one or more trusted secret storage
providers (SSPs) with storing and regulating access to their private
secrets (personal information, encryption keys, etc) on their
behalf. Such secrets can then be used to bootstrap the security of
other data stores via existing cryptographic techniques. Existing
technologies and services can than interface with these SSPs via a
standard interface to access user secrets, and by proxy, the data they
protect as allowed by a user-defined set of access control rules. In
effect, the SSPs are tasked with regulating access to user data by
more traditional feature-oriented third-party services (e.g. Google or
Facebook).

The SSaaS model provides a number of benefits over the existing
practice granting feature-rich third-party services unfettered access
to user data. Such benefits include:

\begin{packed_desc}
\item[No Single Trusted Third Party] \hfill \\ In an SSaaS ecosystem,
  the secret storage provider is separate from the provider of the
  end-user cloud service (e.g. Dropbox, Gmail, etc). Furthermore, a
  user may shard their secrets across multiple SSPs, or even host
  their own SSP. This ensures that a user is not giving any single
  entity control over or unfettered access to their secrets (and by
  proxy, to the data such secrets might protect).
\item[Separation of Duties] \hfill \\ In an SSaaS ecosystem, a user
  selects a secret storage provider on the basis of their trust in
  that provider while selecting a feature service provider on the
  basis of the end-user services they provide. This allows a user to
  optimize each selection individually instead of having to chose a
  single provider on the basis of both trust and feature set, likely
  sacrificing one in favor of the other.
\item[Support for Existing Use Cases] \hfill \\ The SSaaS ecosystem is
  capable of supporting many modern use cases such as sharing data
  with other users, syncing it across multiple personal computing
  devices, or automating access to such data from a variety of
  services. Thus, SSaaS allows users to gain privacy and security
  benefits without having to forgo common and popular use cases.
\end{packed_desc}

\section{Motivating Examples}
\label{chap:intro:example}

As a motivating example, consider the Dropbox cloud file locker
service~\cite{dropbox}. Dropbox provides a service through which users
may upload arbitrary files in order to sync them between multiple
devices and to share them with other users. In order to support this
functionally, Dropbox stores a copy of each user's uploaded files on
the Dropbox servers. This allows Dropbox to download these files to
additional user devices or to share them with others on the user's
behalf.

But how private are a user's files once uploaded to Dropbox? While
Dropbox does encrypt files while they are stored on the Dropbox
servers as well as while they are in transit between the Dropbox
servers and a client machine~\cite{dropbox-security}, Dropbox also
holds a copy of the associated encryption keys, enabling them to
decrypt a user's files whenever they desire. This also means that an
adversary may be able to decrypt user files if they are able to
compromise both Dropbox's data storage servers as well as their key
storage servers. Various governments could also decrypt user files
stored via Dropbox should a court compel Dropbox to provide both the
files and the associated encryption keys. Furthermore, a rogue actor
within Dropbox could leverage their access to all of Dropbox's
infrastructure to access and decrypt user files. Clearly Dropbox's
practice of storing both a user's encrypted files as well as a copy of
the associated encryption keys provides only marginally more security
and privacy than not using encryption at all. And the added security
it does provide is wholly dependent on placing a high degree of trust
in Dropbox to faithfully protect and manage the relevant encryption
keys.

An alternative approach would be for Dropbox to put the user in charge
of encrypting/decrypting files and storing all necessary encryption
keys, ensuring the Dropbox itself never has direct access to
unencrypted user files. While such ``client-side: encryption could
potentially increase the privacy and security of user data in the
event that Dropbox's data stores are compromised, searched, monitored,
or misused, is also has some significant downsides:

\begin{packed_enum}
\item It breaks Dropbox's sharing use case. While user's could still
  share encrypted versions of their files with other users via
  Dropbox, they would also have to exchange the associated encryption
  key out-of-band in order for their college to decrypt and read or
  update any shared file. This nullifies Dropbox's appeal as a simple
  method for sharing files with other users.
\item It complicates Dropbox's syncing use case. Whereas before a
  Dropbox user could bootstrap a new Dropbox client device simply by
  signing into their Dropbox account, users must now both sign into
  their Dropbox account and manually transfer a copy of their
  encryption key to the new device to enable the required encryption
  and decryption operations. This adds an additional step to the
  Dropbox setup process, potentially driving away users.
\item If the user ever losses their encryption keys, they will
  effectively lose access to all of their Dropbox-stored
  files. Similarly, if a user mishandles their keys in a manner that
  allows others to access them, they have effectively negated the
  additional privacy or security benefits client-side encryption
  provides. A user would have to be diligent about ensuring they
  maintain access to their keys via backups, etc, while also ensuring
  their keys do not fall into the wrong hands. Such manual management
  requirements pose a non-trivial burden for most users.
\end{packed_enum}

Thus, the user must chose between the convenience of using Dropbox as
it exists today while also sacrificing a significant degree of privacy
over the data they upload, or the burden of employing traditional
client-side encryption models where the trust they must place in
Dropbox is reduced, but where many core Dropbox features (e.g. simple
file sharing) are no longer feasible to use. Neither of these are
ideal solutions. AN ideal solution must allow the user to leverage the
existing convenience and benefits of using Dropbox while also reducing
the trust they must place in Dropbox itself.

These challenges are not unique to Dropbox. There are many modern
technologies and services that force the user to chose between
convenience and feature set verse privacy and control. For example:

\begin{packed_desc}
\item[Mobile Computing Devices] \hfill \\ Phones, tablets, and laptops
  have become ubiquitous modes of modern computing, storing large
  fractions of our personal data and carrying out many everyday
  computations on our behalf. But these devices, while convenient, are
  also prone to loss, theft, and remote exploitation, exposing the
  data they store and computations they undertake to a range of
  external actors.
\item[Cloud Computing Infrastructure] \hfill \\
  Infrastructure-as-a-Service (IaaS) systems such as Amazon's
  EC2~\cite{amazon-ec2} or Google's Compute
  Engine~\cite{google-compute} are popular mechanisms for hosting
  modern compute services. Unfortunately these services require the
  user to fully trust the backing infrastructure provider and make it
  difficult to deploy security-enhancing systems like full disk
  encryption due to the user's lack of physical server access.
\item[Datacenter Infrastructure] \hfill \\ Modern data centers are
  highly automated. It is thus not acceptable to assume a trusted user
  will be available to enter boot-time passwords in order to bootstrap
  full disk-encryption systems. Thus, using disk encryption systems on
  highly automated servers is difficult, if not impossible, to do
  securely.
\end{packed_desc}

All of these examples share a common deficiency: they force the user
into a position of choosing between desirable feature sets or
desirable security and privacy qualities. This document aims to
quantify such deficiencies and suggest potential resolutions via the
Secret Storage as a Service model.

\section{Goals}
\label{chap:intro:goals}

The goals of the work discussed in this document are threefold:

\begin{packed_item}
\item Quantify and analyze the trust and threat models inherent in
  modern mobile, cloud, and datacenter computing solutions.
\item Provide secret-storage primitives that allow users to minimize,
  manage, and monitor the degree to which they must trust third party
  devices and services with such data.
\item Use such secret-storage primitives to create security and
  privacy enhancing systems well adopted for a range of common and
  desirable use cases.
\end{packed_item}

The remainder of this document is structured as follows:

\begin{packed_desc}
\item[Chapter~\ref{chap:background} - Background:] Presents the
  necessary background knowledge related to this work.
\item[Chapter~\ref{chap:challenges} - Challenges:] Discuses the
  security and privacy challenges of modern use cases.
\item[Chapter~\ref{chap:related} - Related Work:] Describes existing
  work related to enhancing privacy and security.
\item[Chapter~\ref{chap:trust} - Trust Model:] Presents a model for
  analyzing trust and trust violations.
\item[Chapter~\ref{chap:ssaas} - SSaaS:] Presents the Secret Storage
  as a Service model and its merits.
\item[Chapter~\ref{chap:apps} - Applications:] Presents a number of
  potential SSaaS-backed applications and prototypes.
\item[Chapter~\ref{chap:custos} - Custos:] Presents Custos -- a first
  generation SSaaS prototype, as well as its design and implementation.
\item[Chapter~\ref{chap:tutamen} - Tutamen:] Presents Tutamen -- a
  next generation SSaaS prototype, as well as its design and
  implementation.
\item[Chapter~\ref{chap:policy} - Policy:] Discusses the legal and
  policy implications of the SSaaS model.
\item[Chapter~\ref{chap:conclusion} - Conclusion:] Overview of
  contributions and closing thoughts.
\end{packed_desc}

%%  LocalWords:  SSaaS SSPs SSP IaaS Custos Tutamen
