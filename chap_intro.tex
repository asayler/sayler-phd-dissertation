\chapter{Introduction}
\label{chap:intro}

\section{Overview}

Over the last decade, computing has undergone a monumental shift from
locally stored data on a single personal computer to cloud-based data
storage on a multitude of third party servers. This shift has
generated many benefits: sharing data with other users is trivial,
multi-modal communication between users is easy, and compute devices
are largely ephemeral, easily replaced or transitioned between without
any significant overhead or loss of user data. This transition,
however, has a significant side effect: user data is now largely
stored in manners where it is easily accessible to third parties
beyond the user's immediate control. The shift from locally stored and
controlled user data to third-party user data has a number of
increasingly clear consequences, from increased risk of data
compromise by hackers targeting centralized cloud data stores, to
reduced legal protections from government introspection, to the use of
user statistics in ``big-data'' systems capable of ascertaining more
private information than ever before.

The popularity of the cloud model leads one to believe that most users
are willing to trade the privacy and control afforded by local storage
for the convenience and features cloud-based services provide. Never
the less, a 2014 Pew Research study found that over 90\% of American
adults agree that they have lost control over the data they store in
the cloud, 80\% are concerned about how cloud companies are using
their data, and 70\% are concerned about the manner is which the
government might access their data in the
cloud~\cite{pew-privsec14}. Furthermore, the range of both recently
publicized data leaks at large companies
(e.g.~\cite{apple-icloudleak}) as well as ongoing government
intrusions into cloud-based user data stores
(e.g.~\cite{GreenwaldPrism}) has propelled the debate over user
privacy in the age of the cloud to new levels.

The traditional viewpoint holds that users must choose between either
the conveniences the cloud provides or the privacy and security of
locally stored data. I do not feel that this is true. Instead, I
believe that there are mechanisms that can allow users to retain a
high degree of control over how their data is stored, accessed, and
used while still leveraging a variety of modern third-party
services. The key is disentangling the service we wish to use on the
basis of the features they provide from the third parties we must
trust with control over and access to our data.

To achieve this disentanglement, I propose a new paradigm called
Secret Storage as a Service (SSaaS). In an SSaaS ecosystem, a user
designates one or more trusted SSaaS providers (either self hosted or
third party) with storing and regulating access to their private
secrets (personal information, encryption keys, etc) on their
behalf. Existing technologies and services can than interface with
these SSaaS providers via a standard interface to access user secrets
as allowed by a user-defined set of access control rules. I will
discuss several benefits to this arrangement over the existing
practice of selecting third party services on the basis of their
feature set and implicit providing the same providers with unfettered
access to user data. In particular:

\begin{packed_desc}
\item[No Single Trusted Third Party] \hfill \\ In an SSaaS ecosystem,
  the secret storage provider (SSP) is separate from provider of the
  end-user cloud service (e.g. Dropbox, Gmail, etc) or device
  (e.g. Apple, Lenovo, etc). Furthermore, a user may shard their
  secrets across multiple SSPs, or even host their own SSP. This
  ensures that a user is not giving any single entity control over or
  unfettered access to their data.
\item[Separation of Duties] \hfill \\
  In an SSaaS ecosystem, a user selects a secret storage provider on
  the basis of their trust in that provider while selecting a cloud
  service provider on the basis of the end-user features they
  provide. This allows a user to optimize each selection individually
  instead of having to chose a single provider on the basis of both
  trust and feature set, likely having to sacrifice one in favor of
  the other.
\item[Support for Existing Use Cases] \hfill \\ The SSaaS ecosystem is
  capable of supporting many modern use cases such as sharing data
  with other users or syncing it across a number of personal computing
  devices. Thus, SSaaS allows users to gain privacy and security
  benefits without having to forgo common and popular use cases.
\end{packed_desc}

\section{Motivating Examples}

As a motivating example, consider the Dropbox cloud file locker
service~\cite{dropbox}. Dropbox provides a service through which users
may upload arbitrary files in order to sync them between multiple
devices and to share them with other users. In order to support this
functionally, Dropbox stores a copy of each user's uploaded files on
the Dropbox servers. This ensures that Dropbox can provide copies of
the files to new user devices or to other users when asked to sync or
share a file on the user's behalf.

But how Private are a user's files once uploaded to Dropbox? While
Dropbox does encrypt files while they are stored on the Dropbox
servers as well as while they are in transit between the Dropbox
servers and a client machine~\cite{dropbox-security}, Dropbox also
holds a copy of the associated encryption keys, enabling them to
decrypt a user's files whenever they desire. This also means that an
adversary may gain access to the cleartext user files if they are able
to compromise Dropbox's servers. The government could also access
cleartext user files should a court or other entity compel Dropbox to
provide both the files and the associated encryption keys. Clearly
Dropbox's practice of storing both a user's encrypted files as well as
a copy of the associated encryption keys provides only marginally more
security and privacy of the user data then not using encryption at
all.

An alternative approach would be for Dropbox to put the user in charge
of encrypting/decrypting files and storing all necessary encryption
keys, ensuring the Dropbox itself never has direct access to
unencrypted user files. While this form of client-side encryption
could potentially increase the privacy and security of user data in
the event that Dropbox's data stores are compromised, searched,
monitored, or simply misused, is also has some significant downsides:

\begin{packed_enum}
\item It breaks Dropbox's sharing use case. While user's can still
  share encrypted version of their files, they would then have to
  exchange the associated encryption key out of band in order to
  effectively decrypt, read, or update any shared file. This
  essentially nullifies Dropbox's appeal as a simple method for
  sharing files with other users.
\item It complicates Dropbox's syncing use case. Whereas before a
  Dropbox user can bootstrap a new Dropbox client device simply by
  signing into their Dropbox account, users must now both sign into
  their Dropbox account and provide a copy of their encryption
  key/keys in order for the Dropbox client to successfully perform the
  required client-side encryption operations. This adds an additional
  step to the Dropbox setup process, potentially driving away novice
  and lay-users.
\item If the user ever losses their encryption keys, they will
  effectively lose access to all of their Dropbox-stored
  files. Similarly, if a user mishandles their keys in a manner that
  allows others to access them, they have effectively negated the
  additional privacy or security client-side encryption provides. A
  user would have to be diligent about ensuring they maintain access
  to their keys via backups, etc, while also ensuring their keys do
  not fall into the wrong hands. Again, this is a non-trivial burden
  for many users.
\end{packed_enum}

Thus, we're left in a situation where the user must chose between the
convenience of using Dropbox as it exists today while also sacrificing
a significant degree of privacy over the files they upload to Dropbox
or the burden of traditional client-side encryption models where the
trust they must place in Dropbox is reduced, but where many core
Dropbox uses cases (e.g. sharing) are also no longer feasible and the
burden of using Dropbox is significantly increased. Neither of these
are ideal solutions. We would like a solution that allows the user to
leverage the existing convenience and benefits of using Dropbox while
also reducing the trust they must place in the Dropbox corporation (or
those with power over it).

These challenges are not unique to Dropbox. There are many modern
technologies and services that force the user to chose between
convenience of use and feature set or privacy and control of their
data. For example:

\begin{packed_desc}
\item[Mobile Computing Devices] \hfill \\ Phones, tablets, and laptops
  have become ubiquitous modes of modern computing, storing large
  fractions of our personal data and carrying out computations on our
  behave. But these devices, while convenient, are also prone to loss,
  theft, and remote exploitation, exposing the data they store and
  computations they undertake to a range of external actors.
\item[Cloud Computing Infrastructure] \hfill
  \\ Infrastructure-as-a-Service (IaaS) systems such as Amazon's
  EC2~\cite{amazon-ec2} or Google's Compute
  Engine~\cite{google-compute} are popular mechanisms for hosting
  modern compute services. Unfortunately these services require the
  user to fully trust the backing infrastructure provider and make it
  difficult to deploy security-enhancing systems like full disk
  encryption due to the user's lack of physical server access.
\item[User Account Registration] \hfill \\ We're constantly being asked to
  register for services available online. This means proving the same
  identity-confirming personal data to third party after third party
  with little ability to police how this data is stored or used after
  it is provided.
\end{packed_desc}

All of these examples share a common deficiency: they force the user
into a position of choosing between desirable feature sets or desirable
security and privacy qualities. It is this deficiency that I seek to
quantify and resolve.

\section{Background}

This work builds on a number of established topics related to
computing, privacy, and security. I touch on the basics of each in the
remainder of this section. Chapter~\ref{chap:related} touches on
further related work.

\subsection{Cryptography}

Many of the topics discussed in this proposal leverage cryptographic
primitives as the basis of various security and privacy
guarantees. This is largely because it represent a security primitive
that does not rely on trusting specific people, platforms, or systems
in order to securely function. Instead, it requires that we place our
trust in only one thing: the underlying math. This has led to the
proliferation of cryptography as the security primitive on which many
other security features are built.

\subsubsection{Symmetric Cryptography}

Modern cryptographic systems come in two flavors: symmetric
cryptographic and asymmetric cryptography. Symmetric cryptography
algorithms function on the principle that a single ``key'' is used to
both encrypt and decrypt a message. This key must be securely stored,
or if shared, securely exchanged between parties. Anyone with the key
can decrypt the corresponding ciphertext the key was used to
create. The security of a symmetric encryption cipher tends to be
directly related to the length of the encryption key: the longer the
key, the more secure the data encrypted with it is.

While symmetric cryptography algorithms are useful in situations where
a single actor will be both encrypting and decrypting a piece of data
(and thus can hold the required key personally), they pose a major
challenge it situations where multiple parties wish to communicate
securely. In this situation, the parties must find a way to securely
communicate the required symmetric key. In the absence of
cryptographic methods, the only way to securely exchange a key while
avoiding both eavesdroppers and interlopers is to meet in person and
exchange the key manually. The tediousness and lack of practicality of
this task, especially in modern digital communication systems where
multiple actors may be continents apart, led researchers to seek a
better method for secure data exchange in the absence of an inherently
secure communication channel.

\subsubsection{Asymmetric Cryptography}

The major breakthrough in solving this challenge came in 1976 with
Diffie and Hellman's publication of ``New Directions in
Cryptography''~\cite{Diffie1976}. Diffie and Hellman proposed a system
for asymmetric cryptography: a cryptography system in which one key is
used for encryption while a second related key is used for
decryption. When properly designed, it is computationally unfeasible
to derive one of the keys in an asymmetric cryptography system form
the other, allowing a user to publish one of their keys for the public
to consume while keeping their other key private. A member of the
public can then use the user's public key to encrypt a message that
only the holder of the private key will be able to decrypt. If all
members of the public maintain such public/private key pairs, it
becomes possible for any user to send any other user a message that
only the recipient an read without requiring any form of secure
communication channel.

Asymmetric cryptography relies on the existence of ``trapdoor''
functions in order to operate. These functions can be quickly solved
in one direction, but are computationally difficult to reverse without
a special piece of information (e.g. the 'key'). Factoring large
numbers is a classic example of a trapdoor function (and the method on
which many modern public key encryption systems are based). Factoring
large numbers is computationally difficult in cases where some piece
of secret information (e.g. one of the factors) is not known.

Diffie and Hellman proposed a potential implementation of a public key
cryptography system, although the first practical public key crypto
system came a few years latter with the invention of the
RSA~\cite{Rivest1978} algorithm. In addition to public/private key
systems, Diffie and Hellman also proposed a system for joint key
generation where two parties can negotiate a secrete key across an
insecure connection. Like asymmetric cryptography, such a system can
be used to bootstrap secure communications across an insecure
connection by allowing two parties to derive a secret key that can
then be used to facilitate further secret communication using a
symmetric encryption algorithm.

Diffie and Hellman also introduce the concept that asymmetric
encryption can be used to build the two additional core cryptographic
primitives we have come to rely upon: cryptographic verification and
cryptographic authentication. Cryptographic verification (also called
a cryptographic ``signature'') is essentially the reverse of
asymmetric encryption: instead of a member of the public using another
party's public key to encrypt a message that only the target party can
read, the target party uses their private key to encrypt a message
that the public can then decrypt using the target's public key. Since
only the target has access to the private key, and is thus capable of
generating such a message, the target can ``prove'' that a given
message comes from them and that it has not been altered in
transit. Just as asymmetric encryption gives rise to cryptographically
secure signatures, cryptographically secure signatures can give rise
to cryptographically secure authentication. If a user generates a
signed message saying ``I am John'' and sends it to an authentication
server, the server can verify that the message signature is valid by
checking it using John's public key, and thus authenticating John in
the process. The server need only have a list of public keys for each
user. It can then leverage the assertion that only the indented user
has access to the corresponding private key for each of the server's
public keys, and is thus the only one capable of generating a signed
message on the user's behalf, as the basis of user authentication.

\subsubsection{Secret Sharing}

Beyond the rise of public key cryptography, one of the other major
cryptographic breakthroughs of the last fifty years was the invention
of cryptographically secure secret sharing schemes. In particular, Adi
Shamir (the 'S' form ``RSA'') proposed a practical and robust secret
sharing scheme in his 1979 paper ``How to share a
secret''~\cite{Shamir1979}. In this work, Shamir lays out the basics
of what has come to be known as Shamir Secret Sharing: a method for
splitting a piece of information up into two or more pieces in a
manner such that holders of any subset of the pieces cannot infer any
information about the pieces they do not hold or the original
information block as a whole. Shamir Secret Sharing allows a user to
divide a piece of D data into N pieces of which K or more pieces can
be used to recompute the original value of D. A user with fewer than K
pieces, however, has no more information about the value of D than a
user with no pieces. This system provides a highly useful method for
distributing information amongst multiple parties or systems in
situations where no single party or system can be fully trusted.

Shamir Secret Sharing, unlike all known asymmetric encryption
techniques, does not rely on computational complexity as the basis of
its security. Instead, it is fundamentally secure based on information
theory principles. Thus, unlike computationally secure systems such as
RSA, Shamir Secret Sharing can not be broken regardless of the amount
of computational power one posses. Shamir Secret Sharing functions on
the basis of defining a polynomial of degree (K-1) over a finite field
with the D data encoded as the first order-zero term. N points are
then selected from this polynomial and distributed to the
participants. Since K points (but no fewer) will uniquely identify the
original polynomial, and thus allow the derivation of D, K users must
combine their pieces in order to re-compute D.

Shamir Secret Sharing (and related systems) are useful in a wide range
of situations where one needs to distribute trust across multiple
entities. In particular, secret sharing techniques are leveraged in
some cryptographically-based access control systems like that
described in~\cite{Goyal2006}.

\subsection{Usability}

Strong cryptography provides the basis for all of the secure systems
we build today. Unfortunately, strong cryptography has a rather
checkered history when it comes to the usability to securely
cryptographic systems. Since most cryptographic systems merely reduce
the security of a system to the security of the cryptographic keys
protecting a system, how one manages such keys is of the utmost
importance. Manual key management, the de-facto key management
standard for many cryptographic systems, tends to be extremely
challenging for the average user to execute properly: often leading to
security failures that have little to do with the quality of the
cryptography itself.

PGP, one of the longest-running cryptographic security projects, is
known to have major usability challenges, making it largely
incomprehensible to all but the most highly-trained
users~\cite{Whitten1999}. These challenge are largely relates to the
average user's inability to properly mange the various cryptographic
keys required for the proper use of PGP~\cite{green-challenge}. This
has led multiple parties to call for the retirement of PGP and/or to
suggest alternatives~\cite{mailpile, openwhisper, google-endtoend}. It
remains to be seen if any of the purposed alternatives will be able to
provide the level of security offered by PGP while avoiding the
usability pitfalls leveraging PGP traditionally
entails~\cite{green-pgp}.

Similar challenges have been observed with other end-user
cryptographic systems, ranging from secure storage devices to various
communication mediums~\cite{Sweikata2009}. In all cases, properly
obtaining, storing, and controlling access to cryptographic keys and
related cryptographic secrets tends to be a task for which the typical
user is ill-suited. Thus, while cryptographic systems like S/MIME have
seen limited success in the enterprise where a central authority and
staff can handle all key management duties, they have largely failed
for individual computer users~\cite{ramsdell-rfc5751}.

\subsection{Storage}

Data storage has long been one of the core use cases of the digital
age. And the amount of data we generate, process, and store is greater
today then ever before. Tied tightly with data storage mechanism are
the access control mechanisms required to protect the digital data we
store. Digital data storage and access control techniques have morphed
and changed over the last 50 years, and many of these changes have
bearing on the work presented in this proposal.

Early storage and file system technologies often simply neglected
security, lacking robust encryption and access control primitives. The
rise of multi-user operating systems like Unix mandated the creation
of basic file-system access control schemes. Thus we gained the
traditional Unix file access control and permissioning scheme as part
of the virtual file system (VFS) abstraction inherent in all legacy
and modern Unix-like operating systems. This system, however, has a
number of limitations: it supports only a single, basic access control
model (owner, group, R/W/E permissions), it requires a trusted system
for enforcement, and it is strongly coupled to a local system. Systems
like NFS attempt to extend Unix file security semantics beyond the
local machine allowing remote sharing of files, but even these systems
are limited to singular administrative domains and trusted systems.

The Windows NT file system access control model (implemented via the
NTFS file system) extends the flexibility of the traditional Unix
model by adding support for more expressive ACLs. These both allow the
control of additional permissions (e.g. delete, create directory, etc)
as well as more expressive user to permission mappings beyond the
basic owner/group/other Unix model. Furthermore, the Windows NT model
has the ability to delegate user authentication to a local Domain
Controller (DC) capable of centrally managing all users from a single
location. This expands the ability to control file access beyond the
users associated with the local system to the users associated with an
entire administrative domain. Still, this system still has many of the
same limitations as the Unix model: the requirement for a trusted
system for enforcement and the tight coupling to the local
administrative domain.

The rise of the Internet as a reliable and high speed system for
connecting multiple machines across the world as well as the move
toward cloud computing models where computational resources are
outsourced to dedicated providers has increased the demand for secure
storage systems capable of spanning multiple systems and domains. In
order to overcome the limitations posed by traditional file system
security models and accommodate modern multi-user, multi-system use
cases, researchers have proposed a number of newer systems. These
systems try to address one or more of the limitations mentioned
above. Some of them employ cryptographic security models to overcome
the need for a trusted enforcement system. Others are designed to
extend access control semantics beyond the local machine to large
networks or even the global internet. Still others explore the use of
novel access control models more expressive then Unix permissions or
Windows NT ACLs. Many system combine more than one of these approaches
to build a fully featured next generation secure storage system.

\cite{Kher2005} presents a survey of the security models of various
storage systems, sorting such systems into basic networked file
systems, single-user cryptographic file systems, and multi-user
cryptographic file systems. As previously mentioned, basic networked
file systems rely on trusted systems and administrators for the
enforcement of security rules. Examples of such systems include the
Sun Network File System (NFS)~\cite{Sandberg1985}, the Andrew File
Systems (AFS)~\cite{Howard1988}, and the Common Internet File System
(CIFS/SMB)~\cite{microsoft-smb2}. All of these systems are designed
for use within local administrative domains and do not scale well to
global, loosely-coupled distributed systems. To deal with the
scalability issues, researchers have built system like
SFS~\cite{Mazieres1999} or OceanStore~\cite{Kubiatowicz2000} which aim
to reduce the administrative burden of large scale distributed file
systems.

All of these systems, however, rely on some degree of system or
administrator trust. In order to accommodate situations where users do
not wish to place trust on the underlying system or remote servers,
there exist a handful of cryptographically-secure file systems. The
best of these systems offer end-to-end cryptography, meaning that data
is encrypted and decrypted on the client side and the server never has
access to the unencrypted data.  Systems like the Cryptographic File
Systems (CFS)~\cite{Blaze1993} provide basic single-user end-to-end
file encryption. While end-to-end encryption is a powerful security
model for enabling secure storage atop untrusted systems, it does pose
challenges with respect to multi-user, multi-device use cases since it
generally requires that all clients have access to private
cryptographic credentials in order to effectively read or write
files. In order to support both end-to-end encryption and multi-user
scenarios, researchers have proposed multi-user cryptographic storage
systems like SiRiUS~\cite{Goh2003}, Cepheus~\cite{Fu1998}, or
Plutus~\cite{Kallahalla2003}.

Miltchev, et al.~\cite{Miltchev2008} presents a framework for
analyzing the suitability of various distributed file systems for
modern multi-user, multi-domain use cases by analyzing five underlying
file system qualities: authentication, authorization, granularity,
delegation, and revocation. Miltchev, et al. suggests that any secure
large scale file system must successfully address the functionality of
all five of these factors across multiple administrative domains in
order to be an effective multi-user, multi-domain file
system. Miltchev, et al. reach the following conclusions regrading
successful secure multi-user, multi-domain file systems: the use of
public-key cryptography for user authentication is an effective way to
support autonomous delegation, capability-based access control systems
tend to lack support for auditing and accountability, ACL-based access
control systems pose scalability challenges when used across
administrative domains, and revocation and user autonomy are often at
odds.

Beyond traditional local and networked file systems, many users have
turned to cloud-backed storage technologies today. System like
Dropbox~\cite{dropbox} or Google Drive~\cite{google-drive} offer
mechanism for storing arbitrary files on third-party cloud
servers. Such files can either be accessed via a web browser or synced
to a local machine via various client-side utilities. Other cloud
services exchange the traditional file storage model all together in
favor of various object storage abstractions: Amazon's S3 is an
example of such a system~\cite{amazon-s3}. Such systems can either be
used for raw key:object data storage or as a block-device-like basis
for higher level file storage abstractions such as Dropbox or Google
Drive. While system live Dropbox, Google Drive, or Amazon S3 rely on a
centralized, trusted third party storage provider, distributed systems
such as Tahoe-LAFS~\cite{Wilcox-O'Hearn2008} propose an alternate
model where trust in any single system is reduced and storage is
spread across a variety of providers.

\subsection{The Cloud}

The previous 10 years have seen a major shift in the manner in which
users and developers obtain various compute resources. Gone are the
days where one must purchase there own hardware or operate their own
computing systems. Instead, numerous companies are more then happy to
sell you any compute service you require for a pre-established
time-metered rate. This ``Cloud'' computing model significantly lowers
the barrier to entry to those needing to leverage compute resources,
increasing the available compute-backed services and has driving a
vast shifts in the way we use the Internet, store our data, obtain our
entertainment, interact with our friends, and more.

Today cloud-services providers like Amazon, Google, Microsoft,
Rackspace, and IBM globally sell over \$150 billion in cloud services
annually~\cite{Flood2013}. The rapid rise of the cloud computing model
is supported by a number of desirable qualities the cloud can provide
more effectively then traditional self-hosted computing
systems. Namely:

\begin{packed_desc}
\item[OPEX vs CAPEX] \hfill \\ Using cloud-based compute services
  allows companies to shift what are traditionally one-time up-front
  capitol expenditures (CAPEX, e.g. large arrays of servers) to
  regular operational expenditures (OPEX, e.g. a monthly fee). This
  fact leads to a number of potential benefits to cloud
  computing. Where as spinning up traditional compute infrastructure
  requires a large initial investment, cloud compute infrastructure
  can be purchased for as low as a few cents each month. This
  drastically lowers the barrier of entry to those requiring such
  services by eliminating any large up-front costs. Furthermore,
  moving to the cloud makes compute infrastructure a regular,
  predictable expense, easily accounted for when planning
  budgets. Finally, operational expenditures are often more easily
  justified at many organizations without requiring major internal
  review processes, allowing those that purchase compute services via
  the cloud to retain more direct control over how, when, and what
  they purchase. All of these factors collaborate to make the OPEX
  cloud model a more desirable compute purchasing model then the
  traditional CAPEX model.
\item[Flexibility] \hfill \\ The ``pay-for-what-you-need'' cloud
  purchasing model is also far more flexible then the traditional
  in-house computing model. Where as the traditional model requires
  buyers to accurately predict their future compute requirements
  before making the initial purchase, the cloud allows users to scale
  infrastructure as required and without any real need to accurate
  advanced prediction. This makes it far simpler to start a small
  project and grow it into a larger project without requiring any
  large up-front cost or guesswork. Likewise, if a project fails to
  gain traction, it can be efficiently spun down and no one is left
  holding a bunch of no-longer-useful hardware.
\item[Efficiency] \hfill \\ The cloud models offers efficiencies of
  scale in many attributes not available to traditional in-house
  compute users. At the macro-level, it is rare for end-user systems
  to require constant load throughout the day. Instead, services tend
  to see peek usage at certain times of day related to the diurnal
  cycles of their users. Large, international cloud service providers
  can leverage this fact in ways individual hardware operators often
  fail to do. In particular, such providers can ensure their
  underlying hardware is uniformly loaded 24/7/365 by spreading the
  workloads of the underlying tents in different parts of the world
  across globally located infrastructure and systems. This allows most
  large cloud services providers to operate their systems and a level
  and reliable capacity, avoiding the need to oversize systems to
  account for short-lived peak loads. At the micro level, cloud
  providers are generally able to co-locate a large number of compute
  systems in a single data center, allowing them to optimize cooling,
  power, network, and other resources in manners not available to
  smaller in-house server farms. On the power and cooling front, it is
  not uncommon to see cloud data centers that are over $90\%$
  efficient - e.g. a PUE\footnote{Power Usage Effectiveness: The ratio
    of total consumed power to useful IT power.} of
  $\approx1.1$~\cite{google-efficiency}. On the networking front, such
  co-location allows for higher speed, lower energy, data transfers
  between machines. The net result of all these efficiency gains is
  that cloud providers can generally offer compute resources for less
  ongoing cost then what a traditional compute deployment can provide.
\end{packed_desc}


Modern cloud systems come in a range of classes. These classes
generally divide cloud-services up in to the level of abstraction they
provide. The common cloud services in use today include:

\begin{packed_desc}
\item[IaaS:] ``Infrastructure as a Service'' systems describe the
  lowest-level of cloud services. In an IaaS environment, the user is
  provided with remote access to a raw computer - generally a virtual
  machine with a pre-installed operating system - atop which they may
  build and implement their own services. This class of cloud services
  represents the more-or-less direct replacement for the traditional
  in-house compute resources model where a user would start with a raw
  physical machine and build up form there. Amazon
  EC2~\cite{amazon-ec2} and Google Compute
  Engine~\cite{google-compute} are both example of IaaS services.
\item[PaaS:] One step up the stack we have ``Platform as a Service''
  offerings. PaaS systems provide the end user with an environment
  capable of running their code, but abstract away a lot of the lower
  level details of setting up and managing a full IaaS virtual
  machine. This allows users to trade flexibility for simplicity and
  ease of use. Google App Engine~\cite{google-appengine} and
  Heroku~\cite{heroku} are examples of PaaS offerings.
\item[SaaS:] At the top of the cloud service stack, we have ``Software
  as a Service''. This class of service is most generally when lay
  end-users are referring to when they talk about the cloud. SaaS
  offerings represent fully-fledged services that provide some form of
  functionally directly to an end user. Examples of SaaS systems
  include Dropbox~\cite{dropbox}, Gmail~\cite{google-gmail}, and
  Facebook~\cite{facebook}.
\end{packed_desc}

It is not uncommon for one layer of cloud services to be built atop a
lower layer. E.g. an SaaS system might be built atop a PaaS system,
itself built atop an IaaS system.

%% + virtualization
%% + service-oriented architectures

\subsection{Privacy and Security}

%%  LocalWords:  SSaaS SSP Lenovo SSPs IaaS Diffie Hellman's Adi VFS
%%  LocalWords:  Shamir NTFS AFS CIFS SMB SFS OceanStore CFS SiRiUS
%%  LocalWords:  Plutus Miltchev ACL LAFS Rackspace CAPEX OPEX PUE
%%  LocalWords:  PaaS Heroku SaaS
