\chapter{Proposed Work}
\label{chap:planned}

Thus far, I've laid out an overview of the SSaaS model as a means for
limiting third party trust and increasing the security and privacy of
today's computing users across a range of modern use cases. In this
chapter, I propose the remaining work that will be completed for
inclusion in my dissertation. This work all builds on or expands the
work discussed thus far.

\section{Existing Services Surveys}
\label{chap:planned:survey}

The first body of work I propose involves taking the trust framework
presented in Chapter~\ref{chap:trust} and applying it to a variety of
existing services. Such a survey will be useful in further
demonstrating the security and privacy challenges posed by existing
systems. I propose analyzing the trust models of several classes of
existing cloud services, each of which is discussed briefly in the
following subsections.

To conduct these surveys, I will analyze the publicly available
information related to the services in question, including service
provider documentation, independent analyses, and examples of known
security and trust failures. In some cases it may also be necessary to
experiment with the services directly for the purpose of further
exploring how they should be categorized within the proposed trust
analyses framework. These analysis will provide the basis for
improving the design of SSaaS-backed solutions to exhibit more
desirable trust profiles then existing systems without unduly
restricting desirable use cases.

\subsection{Consumer Web Services}

The first set of proposed trust surveys will include an in-depth
analysis of several prototypical consumer-facing cloud
services. Potential candidates for analysis include:

\begin{packed_item}
\item Cloud storage services such as Dropbox~\cite{dropbox},
  SpiderOak~\cite{spideroak}, and BitTorrent
  Sync~\cite{bittorrent-sync}.
\item Social media, communication, and transportation services such as
  Facebook~\cite{facebook}, Gmail~\cite{google-gmail}, and
  Uber~\cite{uber}.
\item Password managers such as LastPass~\cite{lastpass} and
  OnePassword~\cite{onepassword}.
\item Financial services such as banks, finance aggregators like
  Mint~\cite{mint}, and payment services such as Google
  Wallet~\cite{google-wallet} and Apple Pay~\cite{apple-pay}.
\end{packed_item}

\subsection{Developer Web Services}

A second set of proposed trust surveys will focus on common
developer-facing cloud services. Potential candidates for analysis
include:

\begin{packed_item}
\item IaaS providers such as Amazon EC2~\cite{amazon-ec2},
  RackSpace~\cite{rackspace-compute}, and Google Compute
  Engine~\cite{google-compute}.
\item Configuration management solutions such as Chef~\cite{chef} and
  Puppet~\cite{puppet}.
\item Recently-released Key management services such as OpenStack
  Barbican~\cite{openstack-barbican}, Amazon Cloud
  HSM~\cite{amazon-hsm}, and Gezzang zTrustee~\cite{gazzang}.
\end{packed_item}

\section{Implementation}
\label{chap:planned:implement}

Beyond applying the proposed trust analysis framework to a range of
existing services, I propose several implementation advancements of my
existing SSaaS prototype. These advancements will build on the Custos
work discussed in Chapter~\ref{chap:custos}, either improving or
replacing existing components. I discuss the main facets of the
proposed implementation projects below.

\subsection{Second-Generation SSP Server and v2 API}

I propose creating an updated SSP server implementation to replace the
Custos first-generation SSP server presented in
Chapter~\ref{chap:custos}. The existing Custos server is performance
limited due to its lack of high performance backing data store. I
propose a rewrite of the system capable of leveraging production-level
backing stores such as Redis~\cite{redis}. Furthermore, the existing
system would benefit from improvements with respect to its
authentication and authorization subsystems. These systems will be
improved to include simpler interfaces, making it easier to add
plugins to support new access control systems. Finally, I plan to make
the auditing system simpler to interface with third party auditing
systems such as LogRythm~\cite{logrythm} or
Splunk~\cite{splunk}. Collectively, this work will constitute the
creation of the Gen2 SSP prototype.

In addition to updating the SSP server prototype, this will be a good
opportunity to make any necessary changes to the Custos SSaaS
API. Potential API changes will be made with an aim toward ensuring
support for the range of use cases discussed in
Chapter~\ref{chap:apps} and explored in the proposed trust surveys.
Furthermore, API updates will be undertaken to insure compliance with
current REST standards~\cite{ibm-restful} and best
practices~\cite{rest-bestpractices}. Updates will also be made with an
eye to learning from similar recently released key-management APIs
such as those of OpenStack Barbican~\cite{openstack-barbican}. These
changes will result in the creation of the Version 2 Custos SSaaS API.

\subsection{Multi-Party Sharding}

One of the major unexplored parts of the discussed SSaaS model is the
use of multi-party sharding to avoid having to trust a single SSP. In
order to avoid centralizing trust in a single location, such sharding
must be performed and managed from the client-side. This raises a
number of questions and challenges:

\begin{packed_item}
\item How do we properly abstract the secret sharding process so that
  SSaaS clients may leverage it without undue overhead?
\item How do we help users to chose a proper set of SSPs to minimize
  their exposure to the varies multi-SSP trust failures
  (e.g. collusion violations and compelled violations) outlined in
  Chapter~\ref{chap:trust}?\footnote{The answer to this question would
    appear to have parallels with existing failure-resistant storage
    systems such as RAID~\cite{patterson1988} and Ceph's CRUSH
    maps~\cite{ceph-crush}.}
\item How do we best distribute access control rules across multiple
  SSPs in a manner that minimizes SSP collusion as well as management
  overhead?
\end{packed_item}

I propose exploring potential answers to these questions by
integrating multi-party sharding support into an updated version of
the Custos client-side interface libraries. Making multi-SSP sharding
simple to use and straightforward to manage will be a key component of
using it to minimize the degree of trust users must place in
individual SSPs. Having a working implementation that is easy to
integrate with existing SSaaS applications will go a long ways toward
achieving such simplicity and ease of use.

\subsection{Client Applications}

In addition to improving the server-side implementation and adding
more robust client support for multi-party sharding, I propose to
build out several SSaaS client applications similar to those discussed
in Chapter~\ref{chap:apps}. These applications will allow further
demonstration of the potential benefits of the SSaaS model and will
provide usable systems against which various SSaaS concepts may be
evaluated. In particular, I propose focusing on several SSaaS data
encryption clients including:

\begin{packed_desc}
\item[Ceph Encryption Support:] Ceph~\cite{ceph} is a popular
  distributed file system commonly used by IaaS systems such as
  OpenStack~\cite{openstack}. As such, adding SSaaS-backed encryption
  support to Ceph would allow developers to utilize Full Disk
  Encryption mechanisms on cloud-hosted VMs, a use case that is
  currently not possible to achieve in most situations. I propose
  adding SSaaS-backed encryption support to the
  librbd~\cite{ceph-librbd-python, ceph-librbd-src} and/or
  rbd-fuse~\cite{ceph-rbdfuse} components of Ceph, allowing VMs and
  other programs that leverage such components to perform encrypted
  reads and writes to the underlying Ceph storage cluster. In doing
  so, the user will be able to reduce their trust in the Ceph provider
  while still leveraging the benefits of using Ceph-backed VMs in the
  cloud.
\item[File-system Encryption Support:] In addition to Ceph-based FDE,
  it would also be useful to have a workable file-system level
  SSaaS-backed encryption solution. Thus, I propose improving the
  existing FUSE-based EncFS system discussed in
  Chapter~\ref{chap:custos} and/or adding SSaaS support to an existing
  encrypted file system such as eCryptFS~\cite{ecryptfs}. Such a
  system will allow users to leverage SSaaS-backed file encryption
  atop existing file-system-like interfaces such as
  NFS~\cite{sandberg1985}, Dropbox~\cite{dropbox}, or Google
  Drive~\cite{google-drive}.
\item[Cloud Encryption Support:] I also propose exploring the benefits
  that would come from integrating SSaaS encryption support directly
  with one or more cloud storage clients such as
  Dropbox~\cite{dropbox}. Such a cloud-client-integrated solution has
  potential benefits over a general-purpose file-system encryption
  systems due to its the ability to unify the sharing support of both
  the SSaaS ecosystem as well as the backing storage provider in a
  single interface.
\end{packed_desc}

\section{Analysis}
\label{chap:planned:analysis}

The final component of my proposed work will be to analyze the trust,
security, performance, and capabilities of the proposed
implementations discussed above. The analysis will focus on the
improvements and trade-offs offered by SSaaS-backed applications and
the SSaaS model relative to existing and alternate solutions. This
analysis will serve to test on the assertions made in this proposal
and evaluate the degree to which they hold up in real-world systems.

To analyze trust and security, I'll apply the trust framework
discussed in Chapter~\ref{chap:trust} to the proposed client
application implementations. Doing so will also allow comparisons with
existing solutions analyzed as proposed in
Section~\ref{chap:planned:survey}. The trust and security analysis
will also focus on the proposed secret sharding implementation with
the goal of evaluating the suitability of such methods for practical
third-party SSP risk mitigation.

While ideal performance is not a stated goal of the proposed work, I
aim to undertake basic performance measurements such as those
published in~\cite{custos-trios}. The goal of such an analysis will be
to established the performance trade-offs and overhead using SSaaS
methods entails relative to non-SSaaS solutions. Understanding this
overhead will help users and designers to make informed decisions
about where and how to employ SSaaS methods in a manner that maximizes
the security and privacy benefits will minimizing potential
performance losses.

Finally, the capability analyses will take stock of the capabilities
and features of the proposed implementations relative to existing
solution. The aim of such an analysis will be to confirm or refute the
assertions that SSaaS systems can increase privacy and security
without unduly hindering access to popular features. The analysis will
also detail the management and access control capabilities of the
proposed SSaaS implementations.

These analyses, combined with the proposed surveys and implementation
advances, constitute the body of work proposed for the completion of
my doctoral degree. The results of the proposed work will be used to
expand the ideas outlines in this document, leading to the creation of
my full desertion. I hope to complete all such proposed work over the
course of the next year for submission by the end of the Spring 2016
semester.

%%  LocalWords:  SSaaS SpiderOak Uber IaaS RackSpace OpenStack SSP
%%  LocalWords:  Barbican Gezzang zTrustee LastPass OnePassword Redis
%%  LocalWords:  Custos SSPs LogRythm Splunk Ceph's Ceph librbd
%%  LocalWords:  rbd FDE EncFS eCryptFS
