\chapter{Proposed Work}
\label{chap:planned}

Thus far, I've laid out an overview of the SSaaS model as a means for
limiting third party trust and increasing the security and privacy of
today's computing users across a range of modern use cases. In this
chapter, I propose the remaining work that will be completed for
inclusion in my dissertation. This work all builds on or expands the
work discussed thus far.

\section{Existing Services Surveys}

The first body of work I propose involves taking the Trust framework
presented in Chapter~\ref{chap:trust} and applying it to a variety of
existing services of the purpose of analyzing the degree of third
party trust usage of each service entails. Such surveys will be
useful in further demonstrating the current state of the art and
security challenges posed by existing systems. I propose analyzing the
trust models of several classes of existing cloud services, each of
which is discussed briefly in the following subsection.

Such survey will analyze the publicly available information about
currently available services, including both provider-provided
documentation, independent analyses, and examples of known security
failures. In some cases it may also be necessary to experiment with the
services directly for the purpose of further exploring how they should
be categorized within my proposed trust analyses framework. These
analysis will provide the basis designing SSaaS-backed solutions that
exhibit more desirable trust profiles then existing systems.

\subsection{Consumer Web Services}

One proposed trust analysis survey will include an in-depth analysis of
several prototypical end-user facing cloud services. Paternal services
for analysis include:

\begin{packed_item}
\item Cloud storage services such as Dropbox~\cite{dropbox},
  SpiderOak~\cite{spideroak}, and BitTorrent
  Sync~\cite{bittorrent-sync}.
\item Social media, communication, and related services such as
  Facebook~\cite{facebook}, Gmail~\cite{google-gmail}, and
  Uber~\cite{uber}.
\item Password managers such as LastPass~\cite{lastpass} and
  OnePassword~\cite{onepassword}.
\item Financial services such as banks, aggregators like
  Mint~\cite{mint}, and payment services such as Google
  Wallet~\cite{google-wallet} and Apple Pay~\cite{apple-pay}.
\end{packed_item}

\subsection{Developer Web Services}

A second proposed trust analyses survey will focus on common
developer-facing third party services. Potential services for analysis
include:

\begin{packed_item}
\item IaaS providers such as Amazon EC2~\cite{amazon-ec2},
  RackSpace~\cite{rackspace-compute}, and Google Compute
  Engine~\cite{google-compute}.  Sync~\cite{bittorrent-sync}.
\item Traditional configuration management solutions such as
  Chef~\cite{chef} and Puppet~\cite{puppet}.
\item Cloud key management services such as OpenStack
  Barbican~\cite{openstack-barbican}, Amazon Cloud
  HSM~\cite{amazon-hsm}, and Gezzang zTrustee~\cite{gazzang}.
\end{packed_item}

\section{Implementation}

Beyond applying my proposed trust analysis framework to a range of
existing services, I propose several implementation advancements of my
existing SSaaS prototype. These advancements will build on the Custos
work discussed in Chapter~\ref{chap:custos}, either improving or
replacing existing implementations. I discuss the main proposed
implementation projects below.

\subsection{Version 2 SSP Server and API}

I propose creating an updated SSP server implementation to replace the
Custos proof-of-concept SSP server presented in
Chapter~\ref{chap:custos}. The existing Custos server is performance
limited due to it's lack of high performance backing data store. I
propose a rewrite of the system capable of leveraging production-level
backing stores such as Redis~\cite{redis}. Furthermore, the existing
system would benefit form improvements with respect to its
authentication and authorization systems. These systems will be
improved to include simpler interfaces, making it easier to add
plugins to support new access control systems. Finally, I plan to make
the auditing system simpler to interface with third party auditing
systems such as LogRythm~\cite{logrythm} or
Splunk~\cite{splunk}. Collectively, this work will constitute the
creation of the version 2 SSP prototype.

In addition to updating the SSP server prototype, this will be a good
opportunity to make any necessary changes to the Custos SSaaS
API. Potential API changes will be made with an aim toward ensuring
the API is in compliance with current REST
standards~\cite{ibm-restful} and best
practices~\cite{rest-bestpractices}. Updated will also be made with an
eye to learning from similar recently released secret-storage APIs
such as those of OpenStack Barbican~\cite{openstack-barbican}. These
changes will result in the creation of the Version 2 Custos SSaaS API.

\subsection{Multi-Party Sharding}

One of the major unexplored parts of the discussed SSaaS model is the
use of multi-party sharding to avoid having to trust a single SSP. In
order to avoid centralizing trust in a single location, such sharding
must be performed and managed from the client-side. This raises a
number of questions and challenges:

\begin{packed_item}
\item How do we properly abstract the secret sharding process so that
  SSaaS clients may leverage it without undue overhead?
\item How do we help users to chose a proper set of SSPs to minimize
  their exposure to the varied trust failures (e.g. collusion
  violations and compelled violations) outlined in
  Chapter~\ref{chap:trust}? The answer to this question would appear
  to have parallels with existing failure-resistant storage systems
  such as Ceph's CRUSH maps~\cite{ceph-crush}.
\item How do we best distribute access control rules across multiple
  SSPs in a manner that minimizes SSP collusion as well as management
  overhead?
\end{packed_item}

I propose to explore potential answers to these questions by
integrating multi-party sharding support into an updated version of
the Custos client-side interface libraries. Making multi-arty sharding
simple to use and straightforward to manage will be a key component of
using it to minimize the degree of trust users must place in
individual SSPs.

\subsection{Clients}

In addition to improving the server-side implementation and adding
more robust support for multi-party sharding, I propose to build out
several SSaaS client applications similar to those discussed in
Chapter~\ref{chap:apps}. These applications will allow further
demonstration of the potential benefits of the SSaaS model and will
provide usable system against which various SSaaS concepts may be
evaluated. In particular, I propose focusing on several SSaaS data
encryption clients including:

\begin{packed_desc}
\item[Ceph Encryption Support:] Ceph~\cite{ceph} is a popular
  distributed file system commonly used by IaaS systems such as
  OpenStack~\cite{openstack}. As such, SSaaS-backed encryption support
  to Ceph would allow developers to utilize Full Disk Encryption
  mechanisms on cloud-hosted VMs, a use case that is currently quite
  challenging to obtain. I propose adding SSaaS-backed encryption
  support to the librbd~\cite{ceph-librbd-python, ceph-librbd-src}
  and/or rbd-fuse~\cite{ceph-rbdfuse} components of Ceph, allowing VMs
  and other programs that leverage such components to perform
  encrypted reads and writes to the underlying Ceph storage
  cluster. In doing so, the user will be able to reduce their trust in
  the Ceph provider while still leveraging the benefits of using
  Ceph-backed VMs in the cloud.
\item[File-system Encryption Support:] In addition to Ceph-based FDE,
  it would also be useful to have a workable file-system level
  SSaaS-backed encryption solution. Thus, I propose improving the
  existing fuse-based EncFS system discussed in
  Chapter~\ref{chap:custos} and/or adding SSaaS support to an existing
  encrypted file system such as eCryptFS~\cite{ecryptfs}. Such a
  system will allow users to leverage SSaaS-backed file encryption
  atop existing file-system-like interfaces such as
  NFS~\cite{sandberg1985}, Dropbox~\cite{dropbox}, or Google
  Drive~\cite{google-drive}.
\item[Cloud Encryption Support:] I also propose exploring the benefits
  that would come form integrating SSaaS encryption support directly
  with one or more cloud storage clients such as
  Dropbox~\cite{dropbox}. Such a purpose-built solution potentially
  has the ability to smooth over sharing management and multi-device
  sync by unifying the sharing support of both the SSaaS ecosystem as
  well as the backing storage provider in a single interface.
\end{packed_desc}

\section{Analysis}

%%  LocalWords:  SSaaS SpiderOak Uber IaaS RackSpace OpenStack SSP
%%  LocalWords:  Barbican Gezzang zTrustee LastPass OnePassword Redis
%%  LocalWords:  Custos SSPs LogRythm Splunk Ceph's desc Ceph librbd
%%  LocalWords:  rbd FDE EncFS eCryptFS
