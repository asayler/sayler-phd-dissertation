\chapter{Policy Implications}
\label{chap:policy}

The work presented in this document impinges a number of policy
questions in the security and privacy space. A number of these
implications have been alluded to in the previous chapters. This
chapter explores some of these policy questions in greater depth.

\section{Cryptography}
\label{chap:policy:crypto}

Many of the ideas proposed in this document rely on the use of a
variety of cryptographic primitives, including symmetric encryption,
message authentication codes (MAC), asymmetric encryption, asymmetric
authentication, and cryptographic signatures. Such cryptographic
primitives, however, have a checkered legal and policy history, at
least in the United States\footnote{the bulk of this chapter will
  focus on US policy and laws since that is the jurisdiction in which
  the author resides and is most intimately familiar. Similar ideas
  are applicable to other jurisdictions.}. This section provides an
overview of cryptography-related policy concerns.

From the export-control bans of the first crypto wars~\cite{kehl2015}
to the Apple v FBI crypto debate, whether or not strong cryptography
should be available to the general public is a hotly debated
topic. Should the government ever succeed in banning or strongly
curtailing the development, distribution, or use of strong
cryptography, it will become substantially more difficult (if not
impossible) to securely implement SSaaS-related ideas, and by proxy,
the privacy and security enhancements SSaaS systems might
provide. Indeed, any effort that curtails the use or access to strong
cryptography is likely to damage the security of a wide range of digital
systems.

\subsection{A Brief History of Cryptography Regulation}

Strong cryptography is a relatively recent development in the timeline
of human history. Modern digital cryptography didn't really come about
until the end World War Two and the advent of information
theory~\cite{shannon1945}. Prior to that point in time, most
``cryptography'' systems were based on folk-theory, obscure language,
one-time pads, mechanical machines, or other items not soundly rooted
in mathematical theory. Even though the groundwork was laid by the end
of world war two, modern practical digital cryptography systems didn't
really become available until the publication of DES in the mid
1970~\cite{fips46} and the invention of asymmetric cryptography around
the same time~\cite{diffie1976}. The combination of a standardized
symmetric key algorithms coupled with the ability to negotiate and/or
exchange symmetric keys over insecure channels using asymmetric
techniques made digital cryptography a practical tool for securing
communications, the security of which rested on the provable
difficulty of solving certain classes of mathematical problems. From
the late 1970s through the 1980s, digital cryptography remained
largely a tool for governments, militaries, and financial
corporations, and as such was often classified as a ``munition'' and
placed under various export control laws regulating the manner in
which encryption technology could be distributed outside of the United
States.\footnote{The U.S. did allow the export of certain types of
  purposely weak crypto, e.g. crypto with key lengths below 40
  bits. This had the extremely unfortunate effect of encouraging the
  standardization and proliferation of various weak crypto algorithms in a
  wide range of consumer-facing software in order to allow the
  software to be distributed internationally. The proliferation of such
  ``export-grade'' cryptography has come back to bite us today -- many
  applications still contain support for such algorithms, and that
  support makes those products vulnerable to ``downgrade'' attacks that
  trick the software into using this weak crypto, exposing the user to
  attack in the process (e.g. FREAK~\cite{beurdouche2015} and
  Logjam~\cite{adrian2015}.} During this time, cryptographically
secure systems were not readily available for use by individuals not
associated with the aforementioned institutions.

The availability of strong cryptography to the general public begin to
change in the early 1990s with the release of
PGP~\cite{zimmermann-pgp10}. PGP quickly became the first widely
distributed encryption software designed for use by ordinary
individuals. It's Internet-based distribution also quickly drew the
attention of US authorities who viewed it as a violation of the export
control laws banning the export of cryptographic technology with key
length beyond 40-bits. In defiance of such acquisitions, PGP's author
published the entire source code of PGP as a book, the contents of
which could be scanned and compiled by anyone with a
copy~\cite{zimmermann-pgpsource}. This publication forced a discussion
of the First Amendment~\cite{us-constitution-amend1} issues associated
with the distribution of computer source code. Such discussions,
coupled with several court cases~\cite{ninthcir-bernstein,
  sixthcir-junger} in favor of the protection of computer source code
under the First Amendment, led to the US government rolling back
(although not completely eliminating) most of the export control rules
surrounding cryptography by the early 2000s~\cite{kehl2015}.

During this same time, the government tried to standardize a
backdoored encryption system known as the Clipper
Chip~\cite{whitehouse-clipper}. The chip was designed to be embedded
in systems where it would provide ``strong'' encryption via the
Slipjack algorithm while maintaining the government's ability to
decrypt such encryption via the use of escrowed master keys. The
Clipper Chip quickly feel into disfavor, however, when researchers
demonstrated simple methods for bypassing its escrow
mechanisms~\cite{blaze1994}. This was soon followed by discoveries of
weaknesses in the Slipjack algorithm~\cite{biham1998}. The Clipper
chip was never widely adopted at the government largely gave up on
pushing backdoored encryption systems by the start of the 2000s.

After the failure of both the governments export control based
regulations and their pushes for key escrow-based encryption standards
in the 1990s, efforts to control and regulate cryptography were
largely quote through the 2000s and early 2010s. But by the mid 2010s,
however, such efforts again begin to pick up. The Edward Snowden
revelations related to spying abuses by the US National Security
Agency (NSA) in the early 2010s led to a rise in privacy awareness
among the general populace~\cite{pew-privsec14}. This, in turn, led a
push to increase is the sue of strong cryptography across a wide range
of products, from websites~\cite{mozilla-deprecatehttp} to
email~\cite{gmail-blog-encryption} to
smartphones~\cite{ars-ios-encrypt, ars-android-encrypt}. The
associated increase is end-user use of cryptography has led to renewed
calls by law enforcement agencies (most notably the US FBI) to mandate
the use of breakable encryption in any scenario were the government
can get a warrant to access the associated
data~\cite{comey-testimony-encryption}. Such calls have culminated in
the ongoing court battle between Apple and FBI over whether or not the
FBI can use the All Writs Act~\cite{usc-allwrits} to compel Apple to
modify the iPhone operating system to expedite the rate at which the
FBI can guess the pass-code required to decrypt the
phone~\cite{ars-cookvfbi}. Congress has also responded to such calls
by suggesting legislation ranging from the formation of encryption
``study'' committees~\cite{hr4651} to outright bans on the use of
strong encryption~\cite{bennett-burrbill} to bans on such bans of
encryption~\cite{hr4528}. How these court cases and legislative
actions will play out remains to be seen.

\subsection{A Defense of Cryptography}

Government efforts to ban cryptography have been countered with a
variety of legal and policy arguments. These arguments won out during
the efforts to regulate and ban cryptography during the
1990s. Hopefully, they will win out again today.

From a policy perspective, there are a variety arguments against
efforts to ban or regulate various form of cryptography. They tend to
fall along a two standard lines:

\begin{packed_desc}
\item[More Harm Than Good] \hfill \\ As discussed in this document and
  elsewhere~\cite{abelson2015}, cryptography is a critical tool for
  protecting the security and privacy of modern computing
  systems. Thus, any effort aimed at weakening cryptography for the
  purpose of minimizing its effectiveness to ``bad'' actors is just as
  likelihood to minimizing its effectiveness to ``good'' actors that
  rely on cryptography to protect their system and data. Indeed,
  efforts to curtail the use of strong cryptography would severely
  damage the security of financial, communication, and data storage
  systems. Any backdoor available to the US government is an
  additional point of failure that an attacker may use to compromise
  and access a system. Furthermore, any policy of providing the US
  government with special access to cryptographically protected data
  will likely be mirrored by other countries -- many of which may have
  far weaker legal protections against the abuse of such
  powers. E.g. if the US government can demand special access, so can
  China or North Korea -- weakening the security of US citizens abroad
  or anyone wishing to secure their selves and systems from
  potentially repressive regimes. Bans of cryptography thus trade the
  security of a huge number of normal individuals for the insecurity
  of a small number of potentially bad actors -- and that's not a
  trade that's worth making.
\item[Ineffectiveness] \hfill \\ Beyond the fact that bans on
  cryptography are likely to cause more harm than good to society as a
  whole, such bans are widely believed to be largely ineffective. Even
  if the US manages to ban the use of cryptography, cryptographic
  concepts are no longer a niche subject. There are hundreds
  cryptographic products available worldwide, more of half of which
  are developed and distributed outside o after Untied
  States~\cite{schneier2016}. The Internet and modern commerce systems
  make such products widely available to almost anyone. Thus, even if
  the US manages to ban cryptography, those wishing to sue it will
  just obtain cryptography related products elsewhere. Modern
  cryptographic concepts are also widely understood and
  taught~\cite{schneier2010crypto}. There are thus a wide variety of
  individuals capable of creating cryptographic secure systems from
  scratch.\footnote{Although it is not advisable for most individuals
    to attempt to ``roll their own'' cryptography due to the
    likelihood of mistakes and lack of formal vetting and review.} It
  would not be difficult for organizations or individuals wishing to
  employ even banned cryptography concepts to find someone to build
  such systems for them.
\item[Better Options] \hfill \\ Law enforcement organizations (LEOs)
  often cute the ``going dark'' problem as the reason why
  cryptographic restrictions are necessary. ``Going dark'' refer to
  the idea that as the use of cryptography grows the ability to law
  enforcement to monitor criminal activity and prevent or prosecute
  crimes diminishes~\cite{anderson2013}. What this viewpoint fails to
  acknowledge is the fact that this is the golden age of surveillance
  - and law enforcement agencies have access to more data than they
  ever have had before~\cite{swire2011}. Thus, even with the
  widespread use of encryption, LEOs have more access to digital
  criminal data than at any point in human history. As such, there are
  number ways to combat crime without weakening encryption. Such
  mechanisms may require more time and effort than the kind automated
  digital surveillance subverted by the use of encryption, but such
  time and effort is a necessary price to pay to ensure the balance
  between privacy and security. Instead of weakening encryption, law
  enforcement can focused on targeted investigations of specific
  individuals, leveraging the large digital footprints even
  cryptography-employing users leave behind. And even when using
  cryptography, individuals are still prone to a wide range of
  targeted attacks, from efforts to compromise individual systems, and
  subvert any cryptography they may employ in the process to social
  engineering Attucks that are historically very effective against
  even cryptography employing adversaries. AT the end of the day,
  there are numerous tools available to law enforcement to pursue
  criminals. Weakening encryption need not be one of them.
\end{packed_desc}

Beyond policy arguments against regulating or banning cryptography, there
are also a number of legal arguments preventing such efforts. In
particular, any effort to ban cryptography would raise series
constitutional questions, including:

\begin{packed_desc}
\item[1\textsuperscript{st} Amendment] \hfill \\ As previously
  mentioned, bans on the publication of source code,
  cryptography-related or otherwise, raise numerous
  1\textsuperscript{st} Amendment questions. The free speech
  guarantees provided by the 1\textsuperscript{st} Amendment have
  generally been interpreted to apply to source
  code~\cite{ninthcir-bernstein, sixthcir-junger}.\footnote{Indeed,
    this ``code as speech'' interpretation is fairly critical to the
    use of US Copyright Law for the protection of code-related
    intellectual property.} As such, any ban on cryptography source
  code is likely an unconstitutional prior restraint on the author's
  freedom of speech. Traditionally, courts have taken a dim few of
  laws imposing such prior restraint~\cite{scotus-nearvminnesota},
  even in cases where national security my be
  impinged~\cite{scotus-nytvus}. Additionally, efforts to mandate the
  inclusion of a backdoor in certain software are likely
  unconstitutional compelled speech. Just as blocking the publication
  of source code interferes with the author's free speech rights,
  mandating the author publish or included certain components in their
  code is also a violation of their free speech
  rights~\cite{scotus-wooleyvmaynard}. While there have been
  exceptions to such interpretations, it's unlikely the broadness of
  any general purpose cryptography limitations coupled with the likely
  ineffectiveness of such rules would meet the kind of ``narrowly
  tailored speech limits in effective support of a legitimate
  government interest'' requirements such a law would have to overcome
  to meet constitutional muster. Still open to debatable, however, is
  whether or not first amendment protections are limited merely to the
  source code implementing encryption, or whether they apply to
  executable binaries as well. If such protections are limited to raw
  source code, it could effectively prevent the widespread use of
  cryptography to those capable of acquiring and building source
  code. Such an outcome would have a detrimental impact on the general
  lay populace not generally capable of such actions.
\item[2\textsuperscript{nd} Amendment] \hfill \\ One of the more
  creative constitutional defenses of cryptography~\cite{xkcd-504}
  comes via the 2\textsuperscript{nd} Amendment - granting citizens
  the right to ``keep and bear
  arms''~\cite{us-constitution-amend2}. The idea is that since the
  government already has a history of considering cryptography to be a
  weapon, then the second amendment should be used to protect the
  rights of citizens to poses and use cryptography. This theory has
  never been tested in a court, but its not entirely unreasonable to
  believe that cryptography is a necessary component of maintaining a
  ``well regulated militia'' today, and thus should be protected by
  the 2\textsuperscript{nd} Amendment~\cite{scotus-usvmiller}. It is
  also not unreasonable to assume that if the Constitution were being
  written today, cryptography would seem a much more natural fit for
  protection via something like the second amendment then
  firearms. Whether or not this argument will pass judicial scrutiny
  remains to be seen. And even if it does, 2\textsuperscript{nd}
  Amendment are likely to guarantee a weaker set of rights than
  1\textsuperscript{st} Amendment protections since the
  2\textsuperscript{nd} Amendment only protects the right to ``keep
  and bear'' cryptography, where as the first amendment also protects
  its distribution. It may be possible, however, to take a split
  approach to protecting cryptography where things like cryptographic
  source code and research are protected via the 1\textsuperscript{st}
  Amendment whereas binary programs or hardware employing cryptography
  are protected via the 2\textsuperscript{nd} Amendment -- similar in
  mechanism to how the 1\textsuperscript{st} would protect the
  publication of a design of a firearm while the 2\textsuperscript{nd}
  Amendment would protect the possession of the firearm itself.
\item[4\textsuperscript{th} Amendment] \hfill \\ Many law enforcement
  agencies cite the importance of being able to access information for
  which they have secured probable cause warrant as their reason for
  wishing to restrict the use of cryptography. Such opponents fear the
  ability of cryptography to render such lawfully obtained warrants
  useless. Whether or not the 4\textsuperscript{th}
  Amendment~\cite{us-constitution-amend4} can be used to compel a
  cryptography provider to decrypt communications in order to comply
  with a valid warrant remains to be seen. Traditional interpretations
  (including the much malign Third Party
  Doctrine~\cite{thompson-thirdparty}) hold that if a third party
  holds the key necessary to decrypt a user's communication, then they
  can be compelled to provide it to the government -- often without
  even requiring a probable cause warrant at all. But what about cases
  where the third party lacks any ability to decrypt the data? Can
  they still be forced to aid the government in serving a valid
  warrant?  The answer to this question probably hinges on the
  ``reasonableness'' of such a request. The ongoing Apple v. FBI
  litigation may help shed some light on this question. At it's heart,
  it would seem that cryptography provides the ultimate tool for
  protecting an individual's 4\textsuperscript{th} Amendment rights -
  since such technology helps ensure that individual remain ``secure
  in their persons, houses, papers, and effects''. But how best to
  reconcile this protection with a valid ``probable cause'' warrant
  remains is a matter of great debate. And ultimately, especially in
  cases where only the end user posses the information necessary to
  decrypt communication, the answer to this question may have more
  bearing on the 5\textsuperscript{th} Amendment than the
  4\textsuperscript{th}. And it seems unlikely that the need to serve
  a 4\textsuperscript{th} Amendment warrant would ever justify the
  potential violations of the 1\textsuperscript{st} and
  2\textsuperscript{nd} Amendments that any general ban on the use of
  cryptography would entail.
\item[5\textsuperscript{th} Amendment] \hfill \\ Assuming the
  1\textsuperscript{st} and 2\textsuperscript{nd} amendments can
  protects an individuals right to distribute, posses, and use
  cryptography, then it falls on the 5\textsuperscript{th} Amendment
  to protect an individuals right not to be forced to disclose the
  necessary password or encryption key required to decrypt a piece of
  information. The right against self incrimination has long been held
  to protect individuals against compelled testimony -- which should
  include being forced to provide a memorized pass-code or password
  necessary to decrypt a piece of information.\footnote{If the user
    has written down there password, however, or locked their phone
    with a fingerprint, 5\textsuperscript{th} Amendment protections no
    longer apply since the evidence is then physical, not
    testimonial.}  Courts, however, are currently split on whether or
  not the 5\textsuperscript{th} Amendment protects individuals from
  being forced to provide decryption assistance~\cite{usvboucher,
    commonwealthvgelfgatt, usvdoe}. Further complicating matters,
  certain types of cryptography can be designed to decrypt a
  ciphertext to different plaintexts depending on the password or key
  provided. Thus, it is possible that an individual could provide one
  password to decrypt a file to a picture of a cat and another
  password to decrypt a file to a plan for robbing a bank. Additional
  password could provide yet additional contents. In such an
  arrangement, there is no reliable way to know how many different
  plaintexts are hidden within a given ciphertext. Thus, even if the
  government does succeed in compelling an individual to ``decrypt''
  their documents, it is entirely possible they would still wind up
  with a set of files other than the ones they were seeking. And it
  seems unlikely any court would allow the government to continue
  compelling a suspect to provide new pass-phrases until they happened
  to get a decrypted set of tiles that factored their case, especially
  since it's possible no such files exist at all. The
  5\textsuperscript{th} Amendment has historically protect the
  contents of an individuals mind. Cryptography allows an individual
  to expand the contents of their mind by leveraging a small piece of
  memorized information (i.e. a pass-code or praise) to protect much
  larger pieces of offloaded information (e.g. a hard drive). Whether
  or not such an expansion is accessible to the US legal system
  remains to be decided.\footnote{Deciding whether or not such
    expansions are acceptable using cryptography will likely have a
    direct implication on the potential future world where human
    brains are technologically upgradeable. Indeed, there is likely
    little functional difference between using a cryptographically
    protected hard drive couple with a memorize pass-code to expand
    one ``memory'' and installing a bio-mimicking computer chip that
    expands one memory. In a world were you have the technology to
    upgrade the the amount of data you can store (and potentially
    where the government has the technology to ``mind-read'' by
    downloading such data), does the 5\textsuperscript{th} Amendment
    still protect the contents of one's mind? While such questions
    remain largely academic today, it may not be long before courts
    are being asked to decide such questions.}
\end{packed_desc}

There are clear policy reason to avoid curtailing or discouraging the
sue of cryptography, as well as clear legal hurtles to doing
so. Hopefully cryptography remains widely and freely available to all,
since any reduction in access to cryptography would have severe
consequence on a range of security enhancing and privacy preserving
technologies.

\subsection{Cryptography Regulation and SSaaS}



\cite{thompson-thirdparty}


\section{Law Enforcement}
\label{chap:policy:leo}

+ Exceptional Access
+ Rights to Privacy
+ Multi jurisdictional challenges

\section{Secret Storage Providers}
\label{chap:policy:ssp}

+ Data Sharing
+ Liability/Insurance
+ Standards

%%  LocalWords:  SSaaS DES PGP's backdoored Slipjack Snowden NSA LEOs
%%  LocalWords:  Wassenaar nd th
