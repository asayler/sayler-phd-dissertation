\chapter{Policy Implications}
\label{chap:policy}

The work presented in this document impinges a number of policy
questions in the security and privacy space. A number of these
implications have been alluded to in the previous chapters. This
chapter explores some of these policy questions in greater depth.

\section{Cryptography}
\label{chap:policy:crypto}

Many of the ideas proposed in this document rely on the use of a
variety of cryptographic primitives, including symmetric encryption,
message authentication codes (MAC), asymmetric encryption, asymmetric
authentication, and cryptographic signatures. Such cryptographic
primitives, however, have a checkered legal and policy history, at
least in the United States\footnote{the bulk of this chapter will
  focus on US policy and laws since that is the jurisdiction in which
  the author resides and is most intimately familiar. Similar ideas
  are applicable to other jurisdictions.}. This section provides an
overview of cryptography-related policy concerns.

\subsection{Limits on Use of Cryptography}
\label{chap:policy:crypto:limits}

From the export-control bans of the first crypto wars~\cite{kehl2015}
to the Apple v FBI crypto debate, whether or not strong cryptography
should be available to the general public is a hotly debated
topic. Should the government ever succeed in banning or strongly
curtailing the development, distribution, or use of strong
cryptography, it will become substantially more difficult (if not
impossible) to securely implement SSaaS-related ideas, and by proxy,
the privacy and security enhancements SSaaS systems might
provide. Indeed, any effort that curtails the use or access to strong
cryptography is likely to damage the security of a wide range of digital
systems~\cite{Abelson2015}.

Strong cryptography is a relatively recent development in the timeline
of human history. Modern digital cryptography didn't really come about
until the end World War Two and the advent of information
theory~\cite{shannon1945}. Prior to that point in time, most
``cryptography'' systems were based on folk-theory, obscure language,
one-time pads, mechanical machines, or other items not soundly rooted
in mathematical theory. Even though the groundwork was laid by the end
of world war two, modern practical digital cryptography systems didn't
really become available until the publication of DES in the mid
1970~\cite{fips46} and the invention of asymmetric cryptography around
the same time~\cite{diffie1976}. The combination of a standardized
symmetric key algorithms coupled with the ability to negotiate and/or
exchange symmetric keys over insecure channels using asymmetric
techniques made digital cryptography a practical tool for securing
communications, the security of which rested on the provable
difficulty of solving certain classes of mathematical problems. From
the late 1970s through the 1980s, digital cryptography remained
largely a tool for governments, militaries, and financial
corporations, and as such was often classified as a ``munition'' and
placed under various export control laws regulating the manner in
which encryption technology could be distributed outside of the United
States. During this time, cryptographically secure systems were not
readily available for use by individuals not associated with the
aforementioned institutions.

+ Bans on crypto
+ 1A Issues
+ 4A Amendment Issues
+ Wassenaar

\subsection{Proliferation of Cryptography}
\label{chap:policy:crypto:proliferation}


\section{Law Enforcement}
\label{chap:policy:leo}

+ Exceptional Access
+ Rights to Privacy
+ Multi jurisdictional challenges

\section{Secret Storage Providers}
\label{chap:policy:ssp}

+ Data Sharing
+ Liability/Insurance
+ Standards

%%  LocalWords:  SSaaS DES
