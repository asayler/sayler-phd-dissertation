\chapter{Policy Implications}
\label{chap:policy}

The work presented in this document impinges a number of policy
questions in the security and privacy space. A number of these
implications have been alluded to in the previous chapters. This
chapter explores some of these policy questions in greater depth.

\section{Cryptography}
\label{chap:policy:crypto}

Many of the ideas proposed in this document rely on the use of a
variety of cryptographic primitives, including symmetric encryption,
message authentication codes (MAC), asymmetric encryption, asymmetric
authentication, and cryptographic signatures. Such cryptographic
primitives, however, have a checkered legal and policy history, at
least in the United States\footnote{the bulk of this chapter will
  focus on US policy and laws since that is the jurisdiction in which
  the author resides and is most intimately familiar. Similar ideas
  are applicable to other jurisdictions.}. This section provides an
overview of cryptography-related policy concerns.

\subsection{Limits on Use of Cryptography}
\label{chap:policy:crypto:limits}

From the export-control bans of the first crypto wars~\cite{kehl2015}
to the Apple v FBI crypto debate, whether or not strong cryptography
should be available to the general public is a hotly debated
topic. Should the government ever succeed in banning or strongly
curtailing the development, distribution, or use of strong
cryptography, it will become substantially more difficult (if not
impossible) to securely implement SSaaS-related ideas, and by proxy,
the privacy and security enhancements SSaaS systems might
provide. Indeed, any effort that curtails the use or access to strong
cryptography is likely to damage the security of a wide range of digital
systems~\cite{abelson2015}.

\subsubsection{A brief history of Cryptography Regulation}

Strong cryptography is a relatively recent development in the timeline
of human history. Modern digital cryptography didn't really come about
until the end World War Two and the advent of information
theory~\cite{shannon1945}. Prior to that point in time, most
``cryptography'' systems were based on folk-theory, obscure language,
one-time pads, mechanical machines, or other items not soundly rooted
in mathematical theory. Even though the groundwork was laid by the end
of world war two, modern practical digital cryptography systems didn't
really become available until the publication of DES in the mid
1970~\cite{fips46} and the invention of asymmetric cryptography around
the same time~\cite{diffie1976}. The combination of a standardized
symmetric key algorithms coupled with the ability to negotiate and/or
exchange symmetric keys over insecure channels using asymmetric
techniques made digital cryptography a practical tool for securing
communications, the security of which rested on the provable
difficulty of solving certain classes of mathematical problems. From
the late 1970s through the 1980s, digital cryptography remained
largely a tool for governments, militaries, and financial
corporations, and as such was often classified as a ``munition'' and
placed under various export control laws regulating the manner in
which encryption technology could be distributed outside of the United
States. During this time, cryptographically secure systems were not
readily available for use by individuals not associated with the
aforementioned institutions.

The availability of strong cryptography to the general public begin to
change in the early 1990s with the release of
PGP~\cite{zimmermann-pgp10}. PGP quickly became the first widely
distributed encryption software designed for use by ordinary
individuals. It's Internet-based distribution also quickly drew the
attention of US authorities who viewed it as a violation of the export
control laws banning the export of cryptographic technology with key
length beyond 40-bits. In defiance of such acquisitions, PGP's author
published the entire source code of PGP as a book, the contents of
which could be scanned and compiled by anyone with a
copy~\cite{zimmermann-pgpsource}. This publication forced a discussion
of the First Amendment~\cite{us-constitution-amend1} issues associated
with the distribution of computer source code. Such discussions,
coupled with several court cases~\cite{ninthcir-bernstein,
  sixthcir-junger} in favor of the protection of computer source code
under the First Amendment, led to the US government rolling back
(although not completely eliminating) most of the export control rules
surrounding cryptography by the early 2000s~\cite{kehl2015}.

During this same time, the government tried to standardize a
backdoored encryption system known as the Clipper
Chip~\cite{whitehouse-clipper}. The chip was designed to be embedded
in systems where it would provide ``strong'' encryption via the
Slipjack algorithm while maintaining the government's ability to
decrypt such encryption via the use of escrowed master keys. The
Clipper Chip quickly feel into disfavor, however, when researchers
demonstrated simple methods for bypassing its escrow
mechanisms~\cite{blaze1994}. This was soon followed by discoveries of
weaknesses in the Slipjack algorithm~\cite{biham1998}. The Clipper
chip was never widely adopted at the government largely gave up on
pushing backdoored encryption systems by the start of the 2000s.

After the failure of both the governments export control based
regulations and their pushes for key escrow-based encryption standards
in the 1990s, efforts to control and regulate cryptography were
largely quote through the 2000s and early 2010s. But by the mid 2010s,
however, such efforts again begin to pick up. The Edward Snowden
revelations related to spying abuses by the US National Security
Agency (NSA) in the early 2010s led to a rise in privacy awareness
among the general populace~\cite{pew-privsec14}. This, in turn, led a
push to increase is the sue of strong cryptography across a wide range
of products, from websites~\cite{mozilla-deprecatehttp} to
email~\cite{gmail-blog-encryption} to
smartphones~\cite{ars-ios-encrypt, ars-android-encrypt}. The
associated increase is end-user use of cryptography has led to renewed
calls by law enforcement agencies (most notably the US FBI) to mandate
the use of breakable encryption in any scenario were the government
can get a warrant to access the associated
data~\cite{comey-testimony-encryption}. Such calls have culminated in
the ongoing court battle between Apple and FBI over whether or not the
FBI can use the All Writs Act~\cite{usc-allwrits} to compel Apple to
modify the iPhone operating system to expedite the rate at which the
FBI can guess the pass-code required to decrypt the
phone~\cite{ars-cookvfbi}. Congress has also responded to such calls
by suggesting legislation ranging from the formation of encryption
``study'' committees~\cite{hr4651} to outright bans on the use of
strong encryption~\cite{bennett-burrbill} to bans on such bans of
encryption~\cite{hr4528}. How these court cases and legislative
actions will play out remains to be seen.

\subsubsection{A Defense of Cryptography}

+ Bans on crypto
+ 1A Issues
+ 4A Amendment Issues
+ Wassenaar

\subsection{Proliferation of Cryptography}
\label{chap:policy:crypto:proliferation}


\section{Law Enforcement}
\label{chap:policy:leo}

+ Exceptional Access
+ Rights to Privacy
+ Multi jurisdictional challenges

\section{Secret Storage Providers}
\label{chap:policy:ssp}

+ Data Sharing
+ Liability/Insurance
+ Standards

%%  LocalWords:  SSaaS DES PGP's backdoored Slipjack Snowden NSA
%%  LocalWords:  Wassenaar
