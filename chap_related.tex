\chapter{Related Work}
\label{chap:related}

There are a number of related efforts that also seek to fulfill one or
more of the goals of this project as outlined in
Section~\ref{chap:intro:goals}. I discuss some of the more pertinent
of these efforts in this Chapter.

\section{Trust, Threat, and Security Models}
\label{chap:related:models}

Security researchers have developed a number of trust, threat, and
security models for a range of computing systems. Some of these models
are concerned merely with the technical security of the system. Others
expand to account for the many proclivities of human behavior that
have bearings on the security and privacy of computing systems. In all
cases, such models seek to answer the questions:

\begin{packed_item}
\item ``On what assumptions does the security of a given system rest?''
\item ``In what manner can those assumptions be violated?''
\item ``What is the effect of violating such assumptions?''
\end{packed_item}

Security models are a critical part of designing any security
system. The security of a given system is only as good as its weakest
link. Security models provide an analyses if the various links within
and surrounding any given security or privacy enhancing system --
links that must be considered when determining the overall guarantees
provided by a given system as well as the exposure the system to
various threats.

\cite{flowerday2006} provides an analyses if how trust and related
human controls must be considered in the design of any information
security system. It builds on work outside the traditional computer
science space related to theories of trust and privacy in government,
law, and society~\cite{camp2003}. These works lay the foundation for
trust analysis in computing systems and tie the concept of trust in
computing systems to wider theories of trust and privacy as
fundamental concepts.

Beyond trust analyses, there are a number of more general computer
security models and taxonomies that tend to focus on more traditional
technical and operational risks and mitigations. \cite{abbas2005}
discuses security and threat models for Internet-connected
services. \cite{tsipenyuk2005} focuses on common failures in
application-level security. \cite{firesmith2005} dives into a more
abstract analysis of safety and security in computing systems and
provides a framework for analyzing software security risks, potential
harms, and security requirements. Finally, \cite{cebula2010} takes a
comprehensive look at operational cyber-security risks across
technical, human, process, and external risks.

While all of these efforts provided a basis for trust, threat, and
security analysis of computing systems, they do not dive into the
specific intricacies of the third-party trust requirements inherent in
cloud computing systems. This is a deficiency on which I focus in
Chapter~\ref{chap:trust}.

\section{Minimizing Third Party Trust}
\label{chap:related:minimize}

Uneasiness with having to trust third parties has led to a number of
efforts to reduce, limit, and control such trust. Many of these
efforts aim to leverage cryptographic primitives as an alternative to
a trusted third party (TPP). As mentioned in
Section~\ref{chap:background:crypto}, cryptographically-based systems
generally require little to no trust in external systems or parties,
only in the underlying math. Other efforts use information theory
primitives such as the secret sharing schemes (also discussed in
Section~\ref{chap:background:crypto}) to try to limit how much damage
any single TPP can do. Such efforts generally require multiple third
parties to collude to accomplish most attacks. In many cases, these
efforts also explore the trade-offs between the convenience and
capabilities that trusted third party systems can provide and the
security risks of requiring one of more trusted third parties. Various
third-party trust limiting efforts are discussed below.

\subsection{Cryptographic Access Control}

The primary limitation of all of the access control models mentioned
in Section~\ref{chap:background:ac} is their reliance on a trusted
arbiter for enforcement: generally this trusted arbiter is the
operating system or third party system in charge of enforcing the
access control system. This means that the security of these access
control systems is only as good as the security of the system or third
party enforcing them. Thus, if the underlying OS or third party is
compromised, the access control system falls apart. Likewise, anyone
in control of the underlying OS or enforcement system (e.g. an
administrator) automatically gains full control over the access
control system and the ability to bypass it\footnote{E.g. as in the
  case of Edward Snowden's collection of large troves of secret NSA
  data from a facility where he acted as a systems
  administrator.}. This is an acceptable limitations in many
situations, especially those based on a centrally managed system with
existing physical security and administrative safeguards in place. But
in distributed systems or the cloud where physical and administrative
control is not guaranteed, a more robust system that lacks this
``trusted arbiter'' requirement is desirable.

To overcome the need for a trusted enforcement mechanism in access
control systems, researchers have turned to cryptographically-based
access control systems. \cite{goyal2006} and \cite{bethencourt2007}
propose several cryptographically-based access control systems. These
systems are based on the concept of Attribute-Based Encryption
(ABE). ABE schemes allow a user to encrypt a document in a manner such
that the access control rules associated with the document are part of
the encryption process itself. Thus, in order to decrypt/access a
document, a user must satisfy one or more cryptographically guaranteed
access control attributes. \cite{goyal2006} allows user to encrypt
documents that can only be decrypted by users possessing specific
attribute polices encoded in their keys. \cite{bethencourt2007}
extends this concept to allow documents to be encrypted with a full
access control policy embedded in the encryption itself, allowing only
users who's private keys meet a generalized set of requirements to
access the documents. Both these systems allow the construction of
access control systems that do not require any trusted arbiter to
regulate access to objects. Instead, the access control policy is
enforced by the underlying cryptography itself.

Such concepts have not yet been widely deployed in day-to-day use,
possibly because of the complexity and computational overhead of
building and operating such systems. These systems also still push off
the generation, storage, and management of private keys to end users
and administrators, raising many of the same key-management challenges
discussed in Section~\ref{chap:challenges:solutions}.

\subsection{Homomorphic Encryption}

The rise of the cloud as the home to many modern data processing
systems has led to questions about the degree to which third-party
cloud providers should be trusted with access to the data processed on
their infrastructure. Homomorphic encryption systems are designed to
help mitigate this trust. Such systems are designed to perform data
processing operations on encrypted data directly, avoiding the need to
decrypt it and expose the unencrypted data to a third party.

The previous few years have heralded the arrival of numerous
partially-homomorphic encryption systems capable of performing certain
classes of data processing and manipulation operations directly on
encrypted data. System like CryptDB~\cite{popa2011} allow users to
search and query encrypted data directly, allowing the storage of such
databases on untrusted infrastructure. Other systems like
CS2~\cite{kamara2011} provided similar protections and capabilities in
more generic data storage contexts. Such systems are referred to as
partially homomorphic encryption schemes since they only allow a
subclass of all possible operations to be performed on encrypted
data. Such systems have been fielded and shown to be practical today.

Beyond partially homomorphic encryption schemes, fully homomorphic
encryption schemes allow for unrestricted classes of operations to be
performed on encrypted data~\cite{gentry2009}. A number of such
systems have been proposed -- some of which go so far as to propose
the possibility for encrypting computer programs themselves that will
be executed by a fully homomorphic processor~\cite{Breuer2013,
  Brenner2011}. Such system are certainly appealing, but thus far,
building one with suitably low overhead so as to be practical today
has proven elusive.

While homomorphic solutions will likely prove to be part of the
solution to the problem of third party trust, they don't inherently
solve all trust-related problems. In the practical since, the
available homomorphic systems today are limited to only preforming
certain operations. Furthermore, while such systems allow the
processing of data atop untrusted infrastructure, they do not provide
a solution for user cases where you wish to share plain-text data with
other users or otherwise leverage an application that inherently
requires data to be decrypted. As in other cases, they also do not
provide a general solution for the management of the associated
cryptographic keys protecting such operations.

\subsection{Secure Storage}

Beyond data processing, the ability to securely store data atop third
party infrastructure while also minimizing the degree of third party
trust this entails has been and remains a desirable goal. As such, a
number of projects have undertaken efforts aimed at achieving such
protections. Secure data storage is one of the most targeted
applications by security and privacy enhancing systems. And for good
reason -- the ability to securely store data is a critical primitive
for maintaining the security and privacy of our computing systems.

A number of traditional client-server encrypted file systems exist
with the design goal of avoiding server-side
trust~\cite{kher2005}. File systems like
CryptoCache~\cite{jensen2000}, RFS~\cite{dong2011}, and
Plutus~\cite{kallahalla2003} are all designed to allow users to store
files on servers without trusting the server itself. Other systems
like Keypad~\cite{geambasu2011} and CleanOS~\cite{tang2012} are aimed
at securing data atop user devices and protecting that data when a
device is lost or stolen. All of these systems employ various forms of
cryptography to obtain their goals.
 
Beyond traditional client-server secure storage systems, distributed
storage systems like Depot~\cite{mahajan2011},
OceanStore~\cite{kubiatowicz2000}, and Tahoe~\cite{wilcox-o'hearn2008}
are all designed to minimize trust in the underlying storage
infrastructure. Such systems allow users to distribute and store files
across many third-party nodes while also ensuring that the failure,
either accidental or intentional, of a subset of nodes does not result
in the loss, corruption, or exposure of user data. Depot focuses on
data integrity and availability in the presence of untrusted
nodes. Tahoe focus on data integrity and privacy in the presence of
untrusted untrusted nodes. OceanStore has elements of both as well as
a focus on large scale deployments that provide for properties like
data locality to enhance performance.

In many of these systems, however, key management primitives are still
ignored or pushed down to the user, leading to usability issues and
use-case mismatches. My proposed work could be used to extend such
designs to better account for the key management challenges limiting
the use of such systems today.

\section{Enhancing End-User Security}
\label{chap:related:enduser}

The Edward Snowden leaks, as well as the numerous highly publicized
privacy failures of a range of companies from Facebook to Target, have
fueled renewed calls for improved security and privacy enhancing
technologies aimed at end-users. In response to these calls, a number
of organizations have begun offering new classes of security and privacy
enhancing tools. Many of these tools share the goals proposed in this
work -- namely the creation of easy-to-use security and privacy enhancing
tools well adapted for modern usage demands.

\subsection{Communication Tools}

Many of the recent security and privacy enhancing tools have focused
on securing user-to-user communication. Both the contents of and
meta-data associated with such communication has been the focus of
many of the recent NSA leaks~\cite{schneier-metadata}. In reaction,
we've seen the creation of new tools as well as advocacy for the
expanded use of existing secure communion tools.

Email is one of the primary communication mediums today. The
traditional tools of for securing email, OpenPGP~\cite{openpgp} and
its various implementation (e.g. GnuPG~\cite{gnupg} and Symantec
PGP~\cite{pgp}), are not new. But they have seen renewed levels of
interest after Edward Snowden revealed his use of such tools and
advocated for them as effective NSA counter measures. Unfortunately
these tools remain no more usable today then they were 20 years ago,
leading to the multitude of usability issues discussed in
Section~\ref{chap:background:usability}. In response, tools like
Mailpile~\cite{mailpile} and Whiteout~\cite{whiteout} have been
created with the aim of making PGP-based email security more friendly
for the average user. Similarly, Google and Yahoo have both engaged in
efforts aimed and making PGP-like encryption and authentication
systems available to their webmail users~\cite{google-endtoend,
  yahoo-endtoend}. Other systems such as STEED aim to adapt
traditional PGP principles to make the primitives simpler for the
average end-user to deal with~\cite{koch2011}.

But even these recent pushes to re-skin PGP and make it more user
friendly can't overcome many of the fundamental issues with the
system~\cite{green-pgp}. Nor are they well suited for securing another
major form of modern communion: real-time chat. To overcome these
deficiencies, systems like Open Whisper's
TextSecure~\cite{openwhisper} and ChatSecure~\cite{chatsecure} have
been created. Both systems rely on the Off-the-Record secure messaging
protocol to achieve encrypted, authenticated, and
forward-secure\footnote{Forward secrecy is a property of secure
  communication that ensures that the short-term sessions keys used to
  protect individual messages can not be derived from any long-term
  authentication or related keys. In short, it guarantees that prior
  messages can not be decrypted should an adversary manage to
  compromise a long-term user key at an point in the future. Such
  systems are commonly implemented by decoupling session keys from
  user authentication keys, using systems such as Diffie-Hellman key
  derivation~\cite{diffie1976} to generate the former while only
  relying on the latter for the prevention of Man-in-the-Middle
  attacks during the initial setup process.} real-time communication
between two or more parties~\cite{otr-v3, borisov2004,
  goldberg2009}. Such systems aim to make secure real-time
communication as simple as possible, and have shown promise in terms
of popularity and usability. But these systems are not necessarily
well suited to replace PGP since they rely on the real-time nature of
chat communication to attain forward secrecy. Similarly usable and
forward-secure solutions for non-real-time systems like email systems
remain elusive.

While secure communication systems are seeing a lot of development and
showing promise, all of the solutions discussed above are
communication-specific solutions. As such, they are complimentary to
the work presented in this proposal related to general methods for
reducing third party trust and managing secrets and cryptographic
keys.

\subsection{Password Managers}

One of the most common forms of third-party secret storage available
today is that of password storage by end-user password managers. The
well-documented failure of user-chosen passwords as a reliable
security mechanisms~\cite{goodin-bible, goodin-passwords} has led
experts to recommend that users rely instead on a single strong
password that unlocks a password management service capable of storing
unique and random passwords for each website or service they
use~\cite{schneier-passwords, krebs-passwords, brodkin-passman}.

Systems such as LastPass~\cite{lastpass} and
OnePassword~\cite{onepassword} provide users with a hosted platform on
which to manage and store their passwords. These systems utilize a
single-third party for data storage in order to accommodate the
multi-device sync and multi-user sharing use cases presented in
Section~\ref{chap:challenges:usecases}. But in doing so, they require
the user to place a fair amount of trust in the third-party
provider. Such systems do tend to perform client-side encryption and
generally are designed to avoid storing the user's master password in
a recoverable form, mitigating some of the potential for third party
abuse, but at the end of the day, they still require the user to trust
a single third party with their credentials, and by proxy, access to
their online services and accounts.

Password management systems share some of the same goals I propose in
this document -- namely, the storage of end-user secrets in a secure
and easy-to-use manner. Thus, password storage can be viewed as a
special case of the more generic Secret Storage as a Service (SSaaS)
model. In this work, I'll present more versatile and generalized
solutions to the secret storage problem. Such solution could be used
to implement the password manger concept while also decoupling such
implementation from needing to trust any single third party.

\subsection{Storage Tools}

Beyond secure communication and password storage lies the more general
problem of arbitrary secure user storage. As mentioned in
Section~\ref{chap:challenges:usecases}, many user expect such storage
to provide multi-user and multi-device support in addition to raw
secure storage, disqualifying many of the traditional encrypted file
system solutions from the running. Popular storage services like
Dropbox~\cite{dropbox} or Google Drive~\cite{google-drive} are common
today, but these solutions provide the associated third parties with
unfettered access to user data, making them unsuitable for users who
wish their data to remain private.

To overcome issues with systems like Dropbox while still affording
users access to modern file-store amenities, systems such as
SpiderOak~\cite{spideroak}, Tresorit~\cite{tresorit}, and
Wuala~\cite{wuala} have been built with the aim of providing users
with a Dropbox-like alternative that does not require trusting the
storage provider. Such services shift the encryption of data into the
client, helping to ensure the server only ever holds an encrypted copy
of said data. None the less, the confidentially and privacy claims
made by such services have been shown to be dubious~\cite{Wilson2014},
largely due to the fact the a user must still place a large degree of
trust in the backing third party in order to facilitate file sharing
with other users.

As in previous cases, however, these solutions are purpose built for
secure storage, and fail to provide a more general solution for
minimizing third party trust. Furthermore, as mentioned above, these
services still rely on a moderate degree of trust in a single third
party in order to facilitate sharing use cases. The work proposed in
this document could be leveraged to re-implement such systems while
further suppressing the final vestiges of single-party third-party
trust.

\section{Key Management Systems}
\label{chap:related:keymgmt}

As mentioned in Section~\ref{chap:challenges:solutions}, one of the
main hindrances to bootstrapping secure systems\ today is the lack of
a versatile key-management systems capable of supporting modern use
cases while also minimizing third party trust. Such key management
systems are subsets of the larger secret storage problem discussed in
this proposal. I am not, however, the first to recognize key
management as one of the critical components to building flexible and
secure privacy and security enhancing systems.

\subsection{Key Management in Storage Systems}

As discussed previously, many existing secure storage systems fail to
meet the needs of today's users due to the fact that their restrictive
and tightly coupled key management systems fail to accommodate a range
of modern use cases. A number of existing secure file systems have
acknowledged the issues created by tightly coupling key management
with a specific storage solution (or of ignoring key management all
together and forcing the user to deal with securing their keys
themselves). These systems represent a step toward more flexible key
management, and by proxy, more usable secure file storage.

Systems like Plutus~\cite{kallahalla2003} are designed to isolate key
management as a dedicated system component. While this is the first
step toward solving many of the key management problems, Plutus simply
shifts key management into the client, providing some degree of
user-facing key management flexibility, but failing to provide a
standardized and general key management
system. SFS~\cite{mazieres1999} takes this separation a step further
by proposing various standardize mechanisms by which a user might
manage keys and an externally-facing interface for doing so. But even
SFS fails to define a general system for key management, instead
focusing on mechanisms that allow the user to select their own key
management system.

A system such as SFS, however, could be interfaced with a general
secret storage systems such as those proposed in this document. Such
applications of s standardized secret storage systems are discussed
further in Chapter~\ref{chap:apps}.

\subsection{Key Escrow}

Key escrow systems represent on common form of key management system.
key escrow systems exist with the aim of fitting client-side key
management into hierarchical organizational frameworks where upper
level members may need access to subordinate's keys. Such systems have
also been proposed in regulatory environments where law enforcement or
regulatory personal must be able to access certain forms of otherwise
secure data. Escrow systems also prove beneficial in situations where
a key holder would like to hedge against the loss of their copy of the
key without having to fully trust any other single party with a
copy. Such systems often leverage secret-sharing schemes such as
Shamir~\cite{shamir1979} to accomplish there goals.

Early key escrow systems merely provided a trusted third party with a
secondary copy of a given data encryption key (and/or generated data
encryption keys in a manner such that two separate keys could be used
to decrypt any given piece of data)~\cite{denning1996}; e.g. the
infamous NSA-backed Clipper chip. These systems, however, fail the
single-trusted-third-party test, requiring a high degree of user trust
in at least one external entity. To overcome such trust
concentrations, Blaze proposes a distributed key escrow system that
makes key escrow requests transparent and leverages a distributed
consensuses to avoid any single rouge actor missing such a system to
steal a key or gain unauthorized access to end-user
data~\cite{blaze1996}.

While distributed key management systems such as those proposed by
Blaze succeed in managing trust in any singular third party, they fail
to provide for a more general key management system that moves beyond
traditional escrow-based user cases (e.g. key sharing for multi-device
sync or key sharing for document collaboration). Furthermore, the
existing scholarship around key escrow systems fail to explore the
benefits and possibilities inherent in the general and standardized
secret storage systems described in this proposal. The general secrets
storage systems described in this proposal can, however, fulfill many
of the traditional roles of a key-escrow system.

\subsection{Cloud Key Management}

Perhaps the most closely related work to that proposed here is the
various cloud-backed key-management systems introduced over the
previous five years. These systems are a reaction to many of the
developer-facing use cases discussed in
Section~\ref{chap:challenges:usecases}. They aim to make key
management and secure key storage primitives available to developers
building atop cloud platforms.

Rackspace's CloudKeep\cite{cloudkeep-presentation, cloudkeep} (now
OpenStack Barbican~\cite{openstack-barbican}) aims to create a
standardized key management system for use across multiple
applications, avoiding the need to re-implement such systems in each
application. Similar to the SSaaS system proposed in this document,
CloudKeep aims to ease developer burden while increasing the security
of end-user applications by focusing security code in a centralized,
carefully curated system. Similar commercial systems also exist with
the aim of proving developers with turnkey key-management
systems~\cite{gazzang, porticor, rosen2012}. Such systems, however,
lack the generic secret-storage flexibility of the systems discussed
here. They also lack the cross-platform standardized necessary to
incentive some of the trust-preserving behavior discussed in
Chapter~\ref{chap:ssaas}.

Finally, various IaaS platforms have begun offering cloud-based
Hardware Security Module (HSM) solutions. HSMs traditionally refer to
dedicate hardware co-processors capable of storing user cryptographic
keys in a manner that prevents them from ever being extracted. Such
processors are then tasked with performing all necessary cryptographic
operations using such keys on the users behalf. Standardized
interfaces such as PKCS11~\cite{pcks11-standard} exist to allow
programs to communicate and utilize such hardware. The benefits of
such chips lie in their ability to securely store cryptographic keys
and to perform cryptographic operations without exposing such keys to
the shared computer memory space, potentially leaking them in the
process (i.e. as in Heartbleed~\cite{heartbleed}. )Amazon has recently
begun offering a cloud-backed HSM service~\cite{lorier-pkcs11} that
aims to make the benefits of dedicated HSM hardware available in the
virtualized world of the cloud~\cite{amazon-hsm}. Such systems, while
helpful for developers who wish to build HSM-dependent systems that
operate on local or cloud hardware, still require placing a high
degree to trust in a single third party to faithfully operate such
virtual HSMs. The SSaaS work discussed in this proposal could
potentially be used to build similar ``soft'' HSM systems without
requiring single third party trust~\cite{lorier-pkcs11}.

%%  LocalWords:  CryptoCache OceanStore SFS Plutus Custos SSaaS TPP
%%  LocalWords:  Rackspace's CloudKeep Custos's CryptDB CleanOS NSA
%%  LocalWords:  Snowden Mailpile TextSecure ChatSecure LastPass IaaS
%%  LocalWords:  OnePassword Diffie SpiderOak Tresorit Wuala Shamir
%%  LocalWords:  HSMs PKCS Heartbleed Snowden's OpenStack Barbican
