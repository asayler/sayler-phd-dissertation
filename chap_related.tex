\chapter{Related Work}
\label{chap:related}

\subsection{Secure Storage}
Design~\cite{Wulf1974}
RFS~\cite{Dong2011}
Keypad~\cite{Geambasu2011}
CryptoCache~\cite{Jensen2000}
OceanStore~\cite{Kubiatowicz2000}
Tahoe~\cite{Wilcox-O'Hearn2008}

A variety of encrypted file systems exist with the goal of enabling
secure data storage~\cite{Kher2005}. As we have shown, however, many
of these system suffer from the entanglement of key management and the
underlying encryption. We are not the first to recognize the
challenges this entanglement imposes. SFS~\cite{Mazieres1999} and
Plutus~\cite{Kallahalla2003} were designed to separate cryptographic
key management from encrypted data storage, allowing for more flexible
key management in the process. But both SFS and Plutus fail to fully
define a standardized, generic, and flexible external system for
storing and managing keys, making a generic ``Secret Storage as a
Service'' architecture impossible to realize with either system. In
particular, SFS purposely avoids specifying any key management
solution, instead focusing on mechanisms that allow the user to select
their own key management system (e.g. Custos). Plutus provides basic
key management functionality, but it bundles these tightly with the
underlying file system, forcing the user to use both or neither and
preventing the user from selecting dedicated third party SSaaS
providers.

\subsection{Crypto Tricks}
Homomorphic Encryption

\subsection{End-User Security}
\cite{openpgp, pgp, gnupg}
\cite{enigmail, mailpile, Koch2011}
\cite{lastpass, onepassword, apple-icloud}
\cite{schneier-passwords, krebs-passwords, brodkin-passman}.
\cite{spideroak}

Password management systems (e.g.~\cite{lastpass} share some of the
same goals as Custos and can be viewed as a subset of the more generic
SSaaS model. Such systems are designed to enable users to use longer,
less predictable passwords by providing a dedicated system that stores
and fills password feels on a user's behalf. The user must only
remember a single strong password, relying on the password manager to
store and supply long random passwords for all the other services a
user leverages. Custos could be used as the secret storage back end
for a range of existing password manager front ends, providing more
flexible password access control semantics in the process and allowing
the user to select a SSaaS provider of their choice.

\subsection{Key Management and Key Escrow Systems}
\cite{Blaze1996, Denning1996}
\cite{cloudkeep-presentation, cloudkeep}
\cite{gazzang, porticor, Rosen2012}
\cite{amazon-hsm}

Rackspace's CloudKeep~\cite{cloudkeep} aims to create a standardized
key management system for use across multiple applications, avoiding
the need to re-implement such systems in each application. Similar to
Custos, CloudKeep aims to ease developer burden while increasing the
security of end-user applications by focusing security code in a
centralized, carefully curated system. CloudKeep, however, lacks the
generic flexibility and powerful semantics of Custos's authentication
and access control mechanisms.
