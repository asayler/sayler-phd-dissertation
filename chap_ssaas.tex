\chapter{Secret Storage as a Service}
\label{chap:ssaas}

As discussed in Chapter~\ref{chap:challenges}, the reliance on third
parties inherent to many popular use cases today poses a number of
privacy and security related challenges. Fortunately, cryptographic
techniques including encryption and authentication provide the
necessary primitives for building a systems that increases the
security and privacy of users by reducing their exposure to third
party-related abuse. Such systems also provide additional security
outside of the traditional third-party risk model by ensure that
systems such as mobile devices that are prone to loss or theft remain
security even when outside the position of their
owners. Unfortunately, cryptography is not merely ``magic fairy dust''
that we can sprinkle on any security or privacy problem to make it
disappear~\cite{smith2003}. Effectively using cryptographic technique
to secure our data involves ensuring that cryptography-employing
security and privacy enhancing solution are designed securely and
usability.

The key to designing secure and usable cryptographic data security
solutions lies in providing secure and flexible secret storage systems
that can be leveraged to manage and access the associated
cryptographic keys protecting any such system. The failure of
traditional cryptographic systems to account for key management has
led many such systems to be unusable, insecure, and/or ill adapted to
modern user cases. I propose the creation of a standardized Secret
Storage as a Service system designed to provide users with the
necessary tools for managing secrets such as cryptographic keys in a
manner that allows for a range of multi-device and multi-user use
cases and that avoids placing high degrees of trust in signaler third
parties. I present the design and justification of such a system in
this chapter.

\section{Architecture}
\label{chap:ssaas:arch}

Secret Storage as a Service (SSaaS) is a cloud architecture where
users utilize dedicated Secret Storage Providers (SSPs) in addition to
the traditional Feature Providers (FPs) like Amazon, Dropbox, Gmail,
or Facebook. An SSP is tasked with the storage of and access control
to a variety of user secrets from cryptographic keys to personal
data. In the normative case, users will limit themselves to storing
cryptographically protect data on third-party FP servers while storing
the associated cryptographic keys protecting such data with a network
of SSPs. This allows SSPs to be selected on the basis of their
trustworthiness while traditional feature providers can be selected on
the basis of their features. The SSaaS model differs from the
traditional cloud model by allowing users to distribute trust across
multiple third parties (or no third parties at all), ensuring that any
single entity need not be fully trusted, while still enabling many
existing cloud use cases.

\subsection{Stored Secrets}

What kind of secrets do we store with an SSP? I believe that users
should really be able to store arbitrary data with any SSP, allowing
open ended secret storage based applications. That said, the SSP model
works best when secrets stored are not inherently sensitive or
revealing when taken alone. This property helps to mitigate the amount
we must trust each SSP. Thus, storing secrets like cryptographic keys
that alone revel no private user data are generally preferable to
storing privacy revealing secrets like plain-text passwords, social
security numbers, etc. I do, however, leave the decision of what to
store with each SSP up to each user and application, and we'll explore
various types of secret storage in Chapter~\ref{chap:apps}.

Another consideration related to what secrets to store with an SSP is
size. I anticipate SSP-based storage selling at a premium price vs
more traditional cloud storage options like S3~\cite{amazon-s3}. This
is due to the difference in priorities between Secret Storage and
generic cloud storage. An SSP is primarily concerned with safeguarding
user secrets and faithfully implementing a user's access control
specifications for each secret. These priorities may very well incur
additional costs not necessary in a more traditional cloud storage
environments: e.g. the need to locate data centers in specific legal
jurisdictions, a greater emphasis of resistant to compelled violations
via legal representation, etc. Thus, it may be desirable for the user
to minimize the amount of data stored with an SSP as a cost
optimization: again making use cases such as storing cryptographic
keys with an SSP while storing the encrypted data with a more
traditional provider desirable.

For these reasons, I feel that storing cryptographic keys with an SSP
is a common enough use case that some SSPs may specifically optimize
for it. Such ``Key Storage as a Service'' (KSaaS) SSPs represent a
subset of the generic SSaaS model.

\subsection{Secret Storage Providers}

In the SSaaS model, SSPs will offer a standard set of features. These
include a standardized interface, access control primitives, and
auditing capabilities. These features provide the basis of building
privacy-preserving SSaaS-backed applications.

\subsubsection{Secret Storage}

At its core, an SSP provider is offering a key:value data storage
model. Each secret is tagged with a key: a unique identifier,
potentially a UUID~\cite{leach2005} or similar unique ID standard. The
value associated with each key is then the user secret itself, be it a
cryptographic key, personal user data, or arbitrary secret
value. Users are able to query each SSP for the value associated with
a given ID, or to add new secrets to each SSP.

Optionally, SSPs may provide versioning of each id:secret pair. This
may be desirable for use cases where the user wishes to share data
with other users while maintaining the ability to revoke shared access
to future versions of a data set. Such ``lazy
revocation''~\cite{kallahalla2003} capabilities can be built atop
versioning schemes that maintain access control information on a
per-version basis. Such applications are discussed further in
Chapter~\ref{chap:apps}.

I foresee each SSP exposing its key-value secret store via a RESTful
HTTPS-based API. The ubiquity of RESTful interfaces in modern
applications ensures that such an interface will allow simple
communication between a client and the SSP across a wide variety of
platforms. This interface will expose create, read, modify, delete
semantics similar to most existing key:value stores. In fact, I assume
that most SSP implementations will use an off-the-shelf key-value
store as the backend for storing user secrets. I intend for SSPs to
utilize a standard API in order to allow user to interact with
multiple SSPs and transfer their secrets between SSPs.

\subsubsection{Access Control}

The SSP data model associates an access control specification with
each id:secret pair (or in a versioned system, with each
id:version:secret set). This specification governs the manner in which
a given secret can be accessed. Such specification will be provided
and controlled by individual users for each secret they store with the
SSP. The SSP is in charge of faithfully enforcing the access control
specification.

An access control specification control who can create, access,
modify, or delete each secret. In contains information regarding both
authentication (how a user provides they are who they claim to be) as
well as authorization (what permission each authenticated user is
granted). I foresee SSPs offering a standard access control framework
in order to promote interoperability between multiple SSPs.

It's important that the SSP access control model remain
flexible. Since an SSP may be asked to store a variety of secrets in
support of a range of use cases, the access control model must be
expressive enough to avoid artificially limiting the user to specific
use cases or secrets. For example, one use case might require a single
user to satisfy multiple challenges in order to gain access to a
highly sensitive secret while another might require autonomous access
from system possessing an approved token during specific times of day
to access less sensitive secrets used by autonomous processes (e.g. a
data-center server booting an encrypted hard disk or backup systems).

In addition to the key:value storage API operations discussed
previously, an SSP will also expose endpoints for manipulating the
access control parameters associated with each secret as part if the
standard REST API. This interface will allow users to update access
control information to allow data sharing with other users, revoke
prior shared access, etc. This interface will, in turn, require it's
own access control specification to ensure that only approved
modification can be made to any secret's access control rules.

\subsubsection{Auditing}

In addition to access control, each SSP should provide auditing
information related to the manner in which id:secret pair is accessed
or modified. This information is useful to the user in order to
provide additional transparency into the manner in which secrets are
utilized. This auditing can be as simple as basic logging of all
secret access or as complex as a system that automatically analyzes
access patterns to try to detect anomalies that might indicate
potential trust violations.

Auditing information is useful to users for several reasons. In the
event that user data or secrets are ever unintentionally leaked or
compromised, audit information can provide a valuable indication of
the scope of the damage. Furthermore, auditing plays an important role
in allowing users to understand the semantics of access revocations:
since it's unfeasible to revoke access to data another user has
already read, audit information provides a user with the scope of
potential revocable outstanding authorization allowances. E.g. if User
A shares a secret with User B by granting them read access to it via
their SSP, but then decides they'd rather revoke that access, User A
can check the audit logs to determine if User B has yet accessed the
shared secret and thus whether or not guaranteed revocation is even
possible.

As in the prior cases, an SSPs auditing capabilities will need to be
exposed via a REST interface to allow client applications to leverage
audit data. Likewise, audit API functions will need their own set of
access control specifications in order to control who was access to
audit information or the ability to delete that information. As
before, standardizing this interface is desirable from an SSP
interoperability standpoint.

It may also be desirable for SSPs to employ some form of
publicly-verifiable audit trail, similar to the concepts discussed in
~\cite{blaze1996}. Such systems might today be constructed using
block-chain-based primitives such as those available in the BitCoin
crypto-currency network~\cite{Nakamoto2008}. Such systems might
provide more robust variants on the ``warrant cannery'' concept that
has recently become popular amongst a range of third party services
providers as a counter measure against secret warrants, subpoenas, and
court orders~\cite{eff-canary}. The semi-centralized nature of SSPs
make them a desirable point at which to audit and detect unauthorized
access requests for user data.

\subsection{Clients}

While SSPs form one half of the SSaaS architecture, the other half is
formed by clients connecting to and leveraging data from
SSPs. Chapter~\ref{chap:apps} discusses potential SSaaS use cases and
application in detail. I outline some of the basics of SSaaS client
deign here.

An SSaaS client is any system designed to connect to and utilize a
Secret Storage Provider service. Clients communicate with one or more
SSaaS providers via the SSaaS API. Clients can store and retrieve
secrets with each SSP, managing secret access control settings, and
retrieve secret access audit info. Examples of SSaaS client
applications include encrypted file systems, secure communication
systems, dedicated crypto-processing systems, etc. Any system that
stands to benefit from offloading secret storage and management to a
dedicated system, either for the purpose of gaining benefits form the
logically centralized nature of an SSP (e.g. for the purpose of
accessing secrets from multiple devices or for sharing them with
multiple users) or simply to avoid implementing a full secret
management and access control stack locally, is a good candidate for
integration into an SSaaS architecture.

Is the simplest case, each SSaaS client communicates with a single
upstream SSP via a standard SSaaS protocol. The standardized protocol
allows the end-user to specify which SSP they wish to use, but each
secret is stored with only a single SSP. The downside to such
arrangements lies in the fact that storing each secret with only a
single SSP raises both trust and availability concerns (the details of
which are discussed below). To overcome such concerns, I believe it
will often be desirable for each SSaaS client to interact with a
network of several upstream SSPs. In this situation, each secret can
be sharded between multiple SSPs - increasing reliability while
reducing trust.

Such multi-SSP arrangement, however, raise client side management
issues. How are access control rules shared between SSPs? How does the
client keep multiple SSPs in sync? And so on... Answering such
questions is some of the work I propose to undertake in
Chapter~\ref{chap:planned}. But the complexity of such solutions will
likelihood call for the creation of a standardize SSaaS client library
that can be utilized by multiple applications. Such a library avoids
the need for each SSaaS client to reimplement SSaaS communication and
management primitives directly. Building on the idea of a common SSaaS
client library is the idea of a common SSaaS management
application. Many of the SSaaS-related management features,
e.g. setting up access control requirements, checking audit logs,
etc., are common across all SSaaS client applications. It may thus be
desirable to offload such functions to a general SSaaS client
management applications in cases where such functions need not be
tightly coupled with a given SSaaS-backed applications. Such
offloading allows the primary SSaaS-backed applications to focus
purely on the secret storage and retrieval side of the SSaaS API while
dedicated management applications handle all of the access control and
auditing side of the SSaaS API.

\section{Economics}

Part of the the argument in favor of the SSaaS model is
economics. Today, users primary select cloud services on the basis of
their features. When they pay for these services, they're primarily
paying to support the core features such service provide. Privacy and
security, while concerns, are often secondary goals. Furthermore, on
many free cloud services, the ability to harvest user data is the
basis for the service providers business model. These situations
create a number of perverse incentives in terms of a traditional
feature providers goals with respect to user security and
privacy~\cite{anderson2001}. In the first case, the feature provider
simply does not prioritize user security since that's not the primary
basis on which users are choosing to pay for a service. In the
secondary case, a feature provider might actively work to subvert sue
security and privacy in order to further leverage user data to
generate income.

The SSaaS model aims to rectify these issues by introducing the SSP
actors whose primary goal is the protection of user secrets and from
whom users purchase secret storage services on the basis of security
and privacy guarantees first and foremost. Thus, SSaaS's ability to
separate secret storage duties from feature provider duties allows
users to purchase each service on the basis of its associated merits,
avoiding the issues associated with putting features in direct
competition with security and privacy: a competition that security and
privacy have historically lost. Given such separation, independent
markets can form around feature provision and secret protection, all
optimized for the respective priorities of each field.

Beyond the removal of perverse incentives brought about by the
separation of SSPs for FPs, it will also be desirable to encourage a
competitive market amongst multiple SSP providers. In order to achieve
such a market, I advocate for the standardizing of a single
inter-compatible SSaaS protocol. Such a standard protocol gives users
a high degree of mobility between competing SSPs providers, avoiding
vendor lock in. This, in turn, increase the competitive pressures
between providers. In short, the aim of a SSaaS ecosystem is to make
security and privacy trade-able commodities, and to leverage market
powers to price and improve both. A competitive market for secret
storage has a number of security and privacy enhancing benefits
benefits:

\begin{packed_desc}
\item[Reputation] If users can easily switch between SSPs, it forces
  SSPs to compete on the basis of their privacy preserving
  reputations. SSPs who can do a superior job avoid the trust
  violations discussed in Chapter~\ref{chap:trust} can attract more
  users and/or command a higher price for their services. Since SSP's
  reputation is tied solely to their ability to faithfully protect
  user secrets, they will not be able to ``iron over'' any
  privacy-related reputation failings with superior end-user feature
  sets as many traditional cloud providers do today\footnote{As an
    example, consider Facebook's numerous trust
    violations~\cite{goel2014, lomas2014, tsukayama2014} and the fact
    that such violations have had no noticeable impact on the number
    of people using Facebook~\cite{foster2014}. An SSP would enjoy no
    such network benefit from additional services beyond secret
    storage were they to violate user's trust; instead, users would
    simply switch to a new SSP.}.
\item[Multiple Providers] A healthy ecosystem of competing SSPs will
  allow users to select from multiple independent providers over which
  they may shard a single secret. As I'll discuss below, this
  multi-SSP practice provides a number of benefits over relying on a
  single SSP, from additional trust reductions to redundancy.
\item[Cost] As in other competing markets, having a number of
  competing providers will allow the user to select a provider that
  offers the best combination of cost and service.
\end{packed_desc}

The SSaaS model also provides business benefits related to
insurance. Having a dedicated entity in charge of protecting user
secrets (and by proxy, any other data protected by those secrets)
simplifies the process of evaluating risk and liability related to the
projection of sensitive data. Similar to the model used by Certificate
Authorities, SSPs could provide insurance polices to their users to
indemnify them against any loss relating from a trust violation on the
part of the SSP. Likewise, SSPs would underwrite such user-facing
insurance polices with their own insurance polices provided by
independent third party insurers. These insurers would need to perform
independent audits of SSP infrastructure and polices in order to
evaluate trust violation risk, further enticing SSPs to fundamentally
deign themselves for the avoidance of trust violations. Such insurance
benefits are not as readily available in the mixed trust + feature
cloud ecosystem of today since it's far harder to evaluate the privacy
violation risk of a company whose primary objectives are more complex
then secret storage alone. The increasing regulation of user privacy
rights and the penalties associated with violating user privacy
further incentives a system where privacy and security are severable
properties that can be independently regulated, evaluated, and
indemnified - unconnected to the user-facing feature set of a given
third party service.

Likewise, SSaaS provides compliance benefits to users storing highly
regulated data. Instead of having to individually verify that each
end-user cloud service meets the requirements of a specific data
storage regulations, a user could instead simply make sure that their
SSP meets the necessary regulations. Once verified, a single SSP could
be reused with multiple FPs without having to undergo further
compliance verification. SSPs might even proactively obtain specific
compliance certifications to make it easy for their users to comply
with specific regulations, regardless of which feature-proving cloud
service a user wishes to leverage. Such practices would likely prove
highly beneficial in tightly-regulated fields such as health care
(e.g. requiring HIPPA~\cite{hippa} compliance), education
(e.g. requiring FERPA~\cite{ferpa} compliance), and online payments
(e.g. requiring PCI DSS~\cite{pcidss} compliance).

%%  LocalWords:  SSaaS SSPs FPs SSP HTTPS TBD SSaaS's SSP's FP HIPPA
%%  LocalWords:  Shamir's BitCoin FERPA PCI DSS
