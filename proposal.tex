\documentclass[defaultstyle,11pt]{thesis}

\usepackage{array}
\usepackage{epsfig}
\usepackage{url}
\usepackage{float}
\usepackage{caption}
\usepackage{subcaption}
\usepackage{tabu}
\usepackage{tikz-qtree}
\usepackage{fancyvrb}

\usepackage{listings}
\lstset{
  language={},
  basicstyle=\footnotesize,
  numbers=left,
  numberstyle=\tiny,
  stepnumber=1,
  numbersep=5pt,
  showspaces=false,
  showstringspaces=false,
  showtabs=false,
  tabsize=4,
  captionpos=b,
  breaklines=true,
  breakatwhitespace=false,
  frame=single,
  frameround=tttt
}
\lstdefinelanguage{JavaScript}{
  keywords={break, case, catch, continue, debugger, default, delete,
    do, else, finally, for, function, if, in, instanceof, new, return,
    switch, this, throw, try, typeof, var, void, while, with, true,
    false, null},
  morecomment=[l]{//},
  morecomment=[s]{/*}{*/},
  morestring=[b]',
  morestring=[b]'',
  sensitive=true
}

\clubpenalty = 10000
\widowpenalty = 10000

%%%%%%%%%%%%   All the preamble material:   %%%%%%%%%%%%

\title{Securing and Managing Trust in Modern Computing Applications}

\author{Andy}{Sayler}

\otherdegrees{B.S.E.E., Tufts University, 2011 \\
              M.S.C.S., University of Colorado, 2013}

\degree{Doctor of Philosophy}   %  #1 {long descr.}
	   {Dissertation Proposal}  %  #2 {short descr.}

\dept{Department of}		    %  #1 {designation}
     {Computer Science}         %  #2 {name}

\advisor{Prof.}                 %  #1 {title}
        {Dirk Grunwald}         %  #2 {name}

\reader{John Black}             %  2nd person to sign thesis
\readerThree{Eric Keller}       %  3rd person to sign thesis
\readerFour{Blake Reid}         %  4th person to sign thesis
\readerFive{Sangtae Ha}         %  5th person to sign thesis

\abstract{

  \OnePageChapter

  Today, users routinely push vast troves of data into a myriad of
  cloud services, from cloud storage platforms to cloud-based email
  clients to social networks. Unfortunately, controlling this data
  once it leaves a user's hands and passes into the cloud is an
  extremely difficult process. The user is often forced to choose
  between leveraging modern cloud services (and the high degree of
  trust that entails) and maintaining privacy and control of their own
  data. We feel these two choice represent a false dichotomy that can
  be resolved by disentangling the security and access control
  mechanisms for cloud-stored data from the cloud service providers
  that utilize and act on this data. We propose a new class of cloud
  service, Secret Storage as a Service (SSaaS), that allows users to
  reduce the trust they must place in individual third parties by
  storing sensitive user secrets (e.g. encryption keys) with a
  dedicated set of Secret Storage Providers while only storing less
  sensitive user data with the feature providers whose cloud-based
  services they wish to leverage.

  We introduce a framework for evaluating the degree to which we must
  trust various third parties in the cloud, as well as the mechanisms
  by which this trust can be violated, potentially resulting in a loss
  of user privacy. We explore the problems with the traditional cloud
  trust model and discuss ways in which SSaaS helps alleviate these
  issues. We discuss the qualities a desirable SSaaS ecosystem would
  have, present an SSaaS prototype, and propose several applications
  where the SSaaS model can significantly increase user privacy while
  also support a variety of modern cloud use cases.

}

%% \dedication[Dedication]{
%%   TBD
%% }

%% \acknowledgements{
%%   \OnePageChapter
%%   TBD
%% }

%\ToCisShort % use this only for 1-page Table of Contents

\LoFisShort	% use this only for 1-page Table of Figures
% \emptyLoF	% use this if there is no List of Figures

\LoTisShort	% use this only for 1-page Table of Tables
% \emptyLoT	% use this if there is no List of Tables

%%%%%%%%%%%%%%%%%%%%%%%%%%%%%%%%%%%%%%%%%%%%%%%%%%%%%%%%%%%%%%%%%
%%%%%%%%%%%%%%%       BEGIN DOCUMENT...         %%%%%%%%%%%%%%%%%
%%%%%%%%%%%%%%%%%%%%%%%%%%%%%%%%%%%%%%%%%%%%%%%%%%%%%%%%%%%%%%%%%

\begin{document}

\input macros.tex

%\input chap_intro.tex
\chapter{Introduction}
\label{chap:intro}

\section{Overview}
\section{Motivating Example}
\section{Background}

%\input chap_challenges.tex
\chapter{Challenges to Privacy and Security}
\label{chap:challenges}

\section{Modern Use Cases}
\section{Failures of Traditional Solutions}
\section{Need for New Solutions}

%\input chap_trust.tex
\chapter{An Issue of Trust}
\label{chap:trust}

\section{Analyses Framework}
\section{Traditional Model}
\section{SSaaS Model}

%\input chap_ssaas.tex
\chapter{Secret Storage as a Service}
\label{chap:ssaas}

\section{Architecture}
\subsection{Stored Secrets}
\subsection{Secret Storage Providers}
\subsubsection{Secret Storage}
\subsubsection{Access Control}
\subsubsection{Auditing}
\subsection{Clients}
\section{Economics}
\section{Security and Trust}

%\input chap_custos.tex
\chapter{Custos: An SSaaS Prototype}
\label{chap:custos}

\section{Architecture}
\section{Protocol}
\section{Implementation}

%\input chap_apps.tex
\chapter{Applications and Use Cases}
\label{chap:apps}

\section{Storage}
\section{Communication}
\section{Authentication}

%\input chap_related.tex
\chapter{Related Work}
\label{chap:related}

%\input chap_planned.tex
\chapter{Planned Work}
\label{chap:planned}

\section{Surveys}
\section{Implementations}
\section{Analysis}

%\input chap_conclusion.tex
\chapter{Conclusion}
\label{chap:conclusion}


%%%%%%%%%   then the Bibliography, if any   %%%%%%%%%
\bibliographystyle{plain} % or "siam", or "alpha", etc.
\nocite{*}                % list all refs in database, cited or not
\bibliography{refs}       % Bib database in "refs.bib"

%%%%%%%%%   then the Appendices, if any   %%%%%%%%%
\appendix
%\singlespacing
%\input appx_messages.tex
%\input appx_libcustos.tex

\end{document}
